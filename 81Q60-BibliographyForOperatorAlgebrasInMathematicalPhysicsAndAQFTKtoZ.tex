\documentclass[12pt]{article}
\usepackage{pmmeta}
\pmcanonicalname{BibliographyForOperatorAlgebrasInMathematicalPhysicsAndAQFTKtoZ}
\pmcreated{2013-03-22 18:46:24}
\pmmodified{2013-03-22 18:46:24}
\pmowner{bci1}{20947}
\pmmodifier{bci1}{20947}
\pmtitle{bibliography for operator algebras in mathematical physics and AQFT: K-to- Z}
\pmrecord{6}{41566}
\pmprivacy{1}
\pmauthor{bci1}{20947}
\pmtype{Bibliography}
\pmcomment{trigger rebuild}
\pmclassification{msc}{81Q60}
\pmclassification{msc}{03G12}
\pmclassification{msc}{81R50}
\pmclassification{msc}{47C15}
\pmclassification{msc}{81T70}
\pmclassification{msc}{46L35}
\pmclassification{msc}{46L10}
\pmclassification{msc}{46L05}
\pmclassification{msc}{81T60}
\pmclassification{msc}{81T05}
%\pmkeywords{operator algebras in quantum statistical mechanics and AQFTs}

% this is the default PlanetMath preamble. as your knowledge
% of TeX increases, you will probably want to edit this, but
\usepackage{amsmath, amssymb, amsfonts, amsthm, amscd, latexsym}
%%\usepackage{xypic}
\usepackage[mathscr]{eucal}
% define commands here
\theoremstyle{plain}
\newtheorem{lemma}{Lemma}[section]
\newtheorem{proposition}{Proposition}[section]
\newtheorem{theorem}{Theorem}[section]
\newtheorem{corollary}{Corollary}[section]
\theoremstyle{definition}
\newtheorem{definition}{Definition}[section]
\newtheorem{example}{Example}[section]
%\theoremstyle{remark}
\newtheorem{remark}{Remark}[section]
\newtheorem*{notation}{Notation}
\newtheorem*{claim}{Claim}
\renewcommand{\thefootnote}{\ensuremath{\fnsymbol{footnote%%@
}}}
\numberwithin{equation}{section}
\newcommand{\Ad}{{\rm Ad}}
\newcommand{\Aut}{{\rm Aut}}
\newcommand{\Cl}{{\rm Cl}}
\newcommand{\Co}{{\rm Co}}
\newcommand{\DES}{{\rm DES}}
\newcommand{\Diff}{{\rm Diff}}
\newcommand{\Dom}{{\rm Dom}}
\newcommand{\Hol}{{\rm Hol}}
\newcommand{\Mon}{{\rm Mon}}
\newcommand{\Hom}{{\rm Hom}}
\newcommand{\Ker}{{\rm Ker}}
\newcommand{\Ind}{{\rm Ind}}
\newcommand{\IM}{{\rm Im}}
\newcommand{\Is}{{\rm Is}}
\newcommand{\ID}{{\rm id}}
\newcommand{\GL}{{\rm GL}}
\newcommand{\Iso}{{\rm Iso}}
\newcommand{\Sem}{{\rm Sem}}
\newcommand{\St}{{\rm St}}
\newcommand{\Sym}{{\rm Sym}}
\newcommand{\SU}{{\rm SU}}
\newcommand{\Tor}{{\rm Tor}}
\newcommand{\U}{{\rm U}}
\newcommand{\A}{\mathcal A}
\newcommand{\Ce}{\mathcal C}
\newcommand{\D}{\mathcal D}
\newcommand{\E}{\mathcal E}
\newcommand{\F}{\mathcal F}
\newcommand{\G}{\mathcal G}
\newcommand{\Q}{\mathcal Q}
\newcommand{\R}{\mathcal R}
\newcommand{\cS}{\mathcal S}
\newcommand{\cU}{\mathcal U}
\newcommand{\W}{\mathcal W}
\newcommand{\bA}{\mathbb{A}}
\newcommand{\bB}{\mathbb{B}}
\newcommand{\bC}{\mathbb{C}}
\newcommand{\bD}{\mathbb{D}}
\newcommand{\bE}{\mathbb{E}}
\newcommand{\bF}{\mathbb{F}}
\newcommand{\bG}{\mathbb{G}}
\newcommand{\bK}{\mathbb{K}}
\newcommand{\bM}{\mathbb{M}}
\newcommand{\bN}{\mathbb{N}}
\newcommand{\bO}{\mathbb{O}}
\newcommand{\bP}{\mathbb{P}}
\newcommand{\bR}{\mathbb{R}}
\newcommand{\bV}{\mathbb{V}}
\newcommand{\bZ}{\mathbb{Z}}
\newcommand{\bfE}{\mathbf{E}}
\newcommand{\bfX}{\mathbf{X}}
\newcommand{\bfY}{\mathbf{Y}}
\newcommand{\bfZ}{\mathbf{Z}}
\renewcommand{\O}{\Omega}
\renewcommand{\o}{\omega}
\newcommand{\vp}{\varphi}
\newcommand{\vep}{\varepsilon}
\newcommand{\diag}{{\rm diag}}
\newcommand{\grp}{{\mathbb G}}
\newcommand{\dgrp}{{\mathbb D}}
\newcommand{\desp}{{\mathbb D^{\rm{es}}}}
\newcommand{\Geod}{{\rm Geod}}
\newcommand{\geod}{{\rm geod}}
\newcommand{\hgr}{{\mathbb H}}
\newcommand{\mgr}{{\mathbb M}}
\newcommand{\ob}{{\rm Ob}}
\newcommand{\obg}{{\rm Ob(\mathbb G)}}
\newcommand{\obgp}{{\rm Ob(\mathbb G')}}
\newcommand{\obh}{{\rm Ob(\mathbb H)}}
\newcommand{\Osmooth}{{\Omega^{\infty}(X,*)}}
\newcommand{\ghomotop}{{\rho_2^{\square}}}
\newcommand{\gcalp}{{\mathbb G(\mathcal P)}}
\newcommand{\rf}{{R_{\mathcal F}}}
\newcommand{\glob}{{\rm glob}}
\newcommand{\loc}{{\rm loc}}
\newcommand{\TOP}{{\rm TOP}}
\newcommand{\wti}{\widetilde}
\newcommand{\what}{\widehat}
\renewcommand{\a}{\alpha}
\newcommand{\be}{\beta}
\newcommand{\ga}{\gamma}
\newcommand{\Ga}{\Gamma}
\newcommand{\de}{\delta}
\newcommand{\del}{\partial}
\newcommand{\ka}{\kappa}
\newcommand{\si}{\sigma}
\newcommand{\ta}{\tau}
\newcommand{\lra}{{\longrightarrow}}
\newcommand{\ra}{{\rightarrow}}
\newcommand{\rat}{{\rightarrowtail}}
\newcommand{\oset}[1]{\overset {#1}{\ra}}
\newcommand{\osetl}[1]{\overset {#1}{\lra}}
\newcommand{\hr}{{\hookrightarrow}}
\begin{document}
\subsection{Literature on operator algebras in mathematical physics and algebraic quantum field theories (AQFT):}
{\bf Alphabetical order: letters from K to Z.} 
\begin{thebibliography} {299}
\bibitem{KSW}
Kawahigashi, Y., Sato, N. and Wakui, M. (2005). $(2+1)$-dimensional topological quantum field theory from subfactors
and Dehn surgery formula for $3$-manifold invariants. {\em Advances in Mathematics}, {\bf 195}, 165-179.
\bibitem{KLu}
Kazhdan, V. and Lusztig, G. (1994).
Tensor structures arising from affine Lie algebras.
{\em IV, Journal of the American Mathematical Society},
{\bf 7}, 383--453.

\bibitem{Ki}
Kirby, R. (1978).
A calculus of farmed links in $S^3$.
{\em Inventiones Mathematicae}, {\bf 45}, 35--56.

\bibitem{KM}
Kirby, R. and Melvin, P. (1990).
On the $3$-manifold invariants of Witten and Reshetikhin--Turaev.
{\em Inventiones Mathematicae}, {\bf 105}, 473--545.

\bibitem{KO}
Kirillov, A. Jr. and Ostrik, V. (2002).
On $q$-analog of McKay correspondence and ADE classification of
$sl^{(2)}$ conformal field theories.
{\em Advances in Mathematics}, {\bf 171}, 183--227.
math.QA/0101219.

\bibitem{KiR}
Kirillov, A. N. and Reshetikhin, N. Yu. (1988).
Representations of the algebra $U_q(sl_2)$,
$q$-orthogonal polynomials and invariants for links.
{\em Infinite dimensional Lie algebras and groups}, (Ka\v c,
V. G., ed.), Advanced Series in Mathematical Physics, vol. 7,
285--339.

\bibitem{KZ}
Knizhnik, V. and Zamolodchikov, A. (1984).
Current algebra and Weiss-Zumino models in two dimensions.
{\em Nuclear Physics B}, {\bf 247}, 83--103.

\bibitem{KS1}
Kodiyalam, V. and Sunder, V. S. (2001).
Spectra of principal graphs.
{\em International Journal of Mathematics}, {\bf 12}, 203--210.

\bibitem{KS2}
Kodiyalam, V. and Sunder, V. S. (2001).
Flatness and fusion coefficients.
{\em Pacific Journal of Mathematics}, {\bf 201}, 177--204.

\bibitem{KS3}
Kodiyalam, V. and Sunder, V. S. (2001).
Topological quantum field theories from subfactors.
Chapman \& Hall/CRC, Research Notes in Mathematics, {\bf 423}.

\bibitem{Kn}
Kohno, T. (1987).
Monodromy representations of braid groups and Yang--Baxter
equations.
{\em Annales de l'Institut Fourier, Grenoble},
{\bf 37,4}, 139--160.

\bibitem{Kn1}
Kohno, T. (1992).
Topological invariants for $3$-manifolds using
representations of mapping class groups I.
{\em Topology}, {\bf 31}, 203--230.

\bibitem{Kn2}
Kohno, T. (1992).
Three-manifold invariants derived from conformal field theory
and projective representations of modular groups.
{\em International Journal of Modern Physics}, {\bf 6},
1795--1805.

\bibitem{Ko1}
Kosaki, H. (1986).
Extension of Jones' theory on index to arbitrary factors.
{\em Journal of Functional Analysis}, {\bf 66}, 123--140.

\bibitem{Ko2}
Kosaki, H. (1989).
Characterization of crossed product (properly infinite case).
{\em Pacific Journal of Mathematics}, {\bf 137}, 159--167.

\bibitem{Ko3}
Kosaki, H. (1990).
Index theory for type III factors.
in {\em Mappings of operator algebras,
Proceedings of U.S.-Japan Seminar}, (ed. H. Araki and R. V.
Kadison), Birkh\"auser, 129--139.

\bibitem{Ko4}
Kosaki, H. (1993).
Automorphisms in the irreducible decompositions of sectors.
{\em Quantum and non-commutative analysis}, (ed. H. Araki et al.),
Kluwer Academic, 305--316.

\bibitem{Ko5}
Kosaki, H. (1994).
AFD factor of type III$_0$ with many isomorphic
index 3 subfactors.
{\em Journal of Operator Theory}, {\bf 32}, 17--29.

\bibitem{Ko6}
Kosaki, H. (1994).
Some remarks on automorphisms for inclusions of type III factors.
in {\em Subfactors --- Proceedings of the Taniguchi Symposium, Katata ---},
(ed. H. Araki, et al.), World Scientific, 153--171.

\bibitem{Ko7}
Kosaki, H. (1996).
Sector theory and automorphisms for factor-subfactor pairs.
{\em Journal of the Mathematical Society of Japan},
{\bf 48}, 427--454.

\bibitem{KLoi}
Kosaki, H. and Loi, P. H. (1995).
A remark on non-splitting inclusions of type III$_1$
factors.
{\em International Journal of Mathematics}, {\bf 6}, 581--586.

\bibitem{KL}
Kosaki, H. and Longo, R. (1992).
A remark on the minimal index of subfactors.
{\em Journal of Functional Analysis}, {\bf 107}, 458--470.

\bibitem{KMY}
Kosaki, H., Munemasa, A. and Yamagami, S. (1997).
On fusion algebras associated to finite group actions.
{\em Pacific Journal of Mathematics}, {bf 177}, 269--290.

\bibitem{KY}
Kosaki, H. and Yamagami, S. (1992).
Irreducible bimodules associated with crossed
product algebras.
{\em International Journal of Mathematics},
{\bf 3}, 661--676.

\bibitem{Ks}
K\"oster, S. (2002).
Conformal transformations as observables.
{\em Letters in Mathematical Physics},
{\bf 61}, 187--198.

\bibitem{Ks2}
K\"oster, S. (preprint 2003).
Absence of stress energy tensor in CFT$_2$ models.
math-ph/0303053.

\bibitem{Ks3}
K\"oster, S. (2004).
Local nature of coset models.
{\em Reviews in Mathematical Physics},
{\bf 16}, 353--382.
math-ph/0303054.

\bibitem{Ks4}
K\"oster, S. (preprint 2003).
Structure of coset models.
math-ph/0308031.

\bibitem{Kv}
Kostov, I. (1988).
Free field presentation of the $A_n$ coset models on the torus.
{\em Nuclear Physics B}, {\bf 300}, 559--587.

\bibitem{KW}
Kramers, H. A. and Wannier, G. H. (1941).
Statistics of the two dimensional ferromagnet part 1.
{\em Physical Review}, {\bf 60}, 252--262.

\bibitem{Ku}
Kuik, R. (1986).
On the $q$-state Potts model by means of non-commutative algebras.
Ph.D. Thesis Groningen.

\bibitem{KuR}
Kulish, P. and Reshetikhin, N. (1983).
Quantum linear problem for the sine-Gordon equation
and higher representations.
{\em Journal of Soviet Mathematics},
{\bf 23}, 2435--2441.

\bibitem{KAW}
Kuniba, A., Akutsu, Y. and Wadati, M. (1986).
Virasoro algebra, von Neumann algebra
and critical eight vertex SOS model.
{\em Journal of Physics Society of Japan}, {\bf 55}, 3285--3288.

\bibitem{La1}
Landau, Z. (2001).
Fuss-Catalan algebras and chains of intermediate subfactors.
{\em Pacific Journal of Mathematics},
{\bf 197}, 325--367.

\bibitem{La2}
Landau, Z. (2002).
Exchange relation planar algebras.
{\em Journal of Functional Analysis}, {\bf 195}, 71--88.

\bibitem{Lic}
Lickorish, W. (1988).
Polynomials for links.
{\em Bulletin of the American Mathematical Society},
{\bf 20}, 558--588.

\bibitem{Loi1}
Loi, P. H. (1988).
On the theory of index and type III factors.
Thesis, Pennsylvania State University.

\bibitem{Loi2}
Loi, P. H. (1996).
On automorphisms of subfactors.
{\em Journal of Functional Analysis},
{\bf 141}, 275--293.

\bibitem{Loi3}
Loi, P. H. (1994).
On the derived tower of certain inclusions of type
III$_\lambda$ factors of index 4.
{\em Pacific Journal of Mathematics},
{\bf 165}, 321--345.

\bibitem{Loi4}
Loi, P. H. (1994).
Remarks on automorphisms of subfactors.
{\em Proceedings of the American Mathematical Society},
{\bf 121}, 523--531.

\bibitem{Loi5}
Loi, P. H. (1997).
Periodic and strongly free automorphisms on inclusions of
type III$_\lambda$ factors.
{\em International Journal of Mathematics},
{\bf 8}, 83--96.

\bibitem{Loi6}
Loi, P. H. (1998).
A structural result of irreducible inclusions of type
III$_\lambda$ factors.
{\em Proceedings of the American Mathematical Society},
{\bf 126}, 2651--2662.

\bibitem{Loi7}
Loi, P. H. (1998).
Commuting squares and the classification of finite depth
inclusions of AFD type III$_\lambda$ factors, $\lambda\in(0,1)$.
{\em Publications of the RIMS, Kyoto University},
{\bf 34}, 115--122.

\bibitem{Lo}
Loke, T. (1994).
Operator algebras and conformal field theory of the
discrete series representations of Diff$(S^1)$.
{\em Thesis, University of Cambridge}.

\bibitem{Ln1}
Longo, R. (1978)
A simple proof of the existence of modular
automorphisms in approximately finite dimensional
von Neumann algebras.
{\em Pacific Journal of Mathematics}, {\bf 75}, 199--205.

\bibitem{Ln2}
Longo, R. (1979).
Automatic relative boundedness of derivations in
$C^*$-algebras.
{\em Journal of Functional Analysis},
{\bf 34}, 21--28.

\bibitem{Ln3}
Longo, R. (1984).
Solution of the factorial Stone-Weierstrass conjecture.
An application of the theory of standard split $W^*$-inclusions.
{\em Inventiones Mathematicae}, {\bf 76}, 145--155.

\bibitem{Ln4}
Longo, R. (1987).
Simple injective subfactors.
{\em Advances in Mathematics}, {\bf 63}, 152--171.

\bibitem{Ln5}
Longo, R. (1989).
Index of subfactors and statistics of quantum fields, I.
{\em Communications in Mathematical Physics}, {\bf 126}, 217--247.

\bibitem{Ln6}
Longo, R. (1990).
Index of subfactors and statistics of quantum fields II.
{\em Communications in Mathematical Physics}, {\bf 130}, 285--309.

\bibitem{Ln7}
Longo, R. (1992).
Minimal index and braided subfactors.
{\em Journal of Functional Analysis}, {\bf 109}, 98--112.

\bibitem{Ln8}
Longo, R. (1994).
A duality for Hopf algebras and for subfactors I.
{\em Communications in Mathematical Physics},
{\bf 159}, 133--150.

\bibitem{Ln9}
Longo, R. (1994).
Problems on von Neumann algebras suggested by
quantum field theory.
in {\em Subfactors --- Proceedings of the Taniguchi Symposium, Katata ---},
(ed. H. Araki, et al.),
World Scientific, 233--241.

\bibitem{Ln10}
Longo, R. (1997).
An analogue of the Kac-Wakimoto formula and black hole
conditional entropy.
{\em Communications in Mathematical Physics},
{\bf 186}, 451--479.

\bibitem{Ln11}
Longo, R. (1999).
On the spin-statistics relation for topological charges.
in {\em Operator Algebras and Quantum Field Theory}
(ed. S. Doplicher, et al.), International Press, 661--669.

\bibitem{Ln12}
Longo, R. (2001).
Notes for a quantum index theorem.
{\em Communications in Mathematical Physics},
{\bf 222}, 45--96.

\bibitem{Ln13}
Longo, R. (2003).
Conformal subnets and intermediate subfactors.
{\em Communications in Mathematical Physics},
{\bf 237}, 7--30.
math.OA/0102196.

\bibitem{LRe}
Longo, R. and Rehren, K.-H. (1995).
Nets of subfactors.
{\em Reviews in Mathematical Physics},
{\bf 7}, 567--597.

\bibitem{LRe2}
Longo, R. and Rehren, K.-H. (2004).
Local fields in boundary CFT.
{\em Reviews in Mathematical Physics},
{\bf 16}, 909--960.
math-ph/0405067.

\bibitem{LRe3}
Longo, R. and Rehren, K.-H. (preprint 2007).
How to remove the boundary.
arXiv:0712.2140.

\bibitem{LRo}
Longo, R. and Roberts, J. E. (1997).
A theory of dimension.
{\em $K$-theory}, {\bf 11}, 103--159.

\bibitem{LX}
Longo, R. and Xu, F. (2004).
Topological sectors and a dichotomy in conformal field theory.
{\em Communications in Mathematical Physics}, {\bf 251}, 321--364.
math.OA/0309366.

\bibitem{Mar}
Markov, A. (1935).
\"Uber de freie Aquivalenz geschlossener Z\"opfe.
{\em Rossiiskaya Akademiya Nauk, Matematicheskii Sbornik}, {\bf 1}, 73--78.

\bibitem{Masuda}
Masuda, T. (1997). An analogue of Longo's canonical endomorphism for
bimodule theory and its application to asymptotic inclusions.
{\em International Journal of Mathematics},
{\bf 8}, 249--265.

\bibitem{Masuda2}
Masuda, T. (1999). Classification of actions of discrete amenable groups on
strongly amenable subfactors of type $III_\lambda$.
{\em Proceedings of the American Mathematical Society},
{\bf 127}, 2053--2057.

\bibitem{Masuda3}
Masuda, T. (1999). Classification of strongly free actions of discrete amenable groups on
strongly amenable subfactors of type $III_0$. {\em Pacific Journal of Mathematics},
{\bf 191}, 347--357.

\bibitem{Masuda4}
Masuda, T. (2000). Generalization of Longo-Rehren construction to subfactors of
infinite depth and amenability of fusion algebras.
{\em Journal of Functional Analysis}, {\bf 171}, 53--77.

\bibitem{Masuda5}
Masuda, T. (2001). Extension of automorphisms of a subfactor to the symmetric
enveloping algebra. {\em International Journal of Mathematics},
{\bf 12}, 637--659.

\bibitem{Masuda6}
Masuda, T. (in press). Classification of approximately inner actions of discrete
amenable groups on strongly amenable subfactors.
{\em International Journal of Mathematics},

\bibitem{Masuda7}
Masuda, T. (2003).Notes on group actions on subfactors.
{\em Journal of the Mathematical Society of Japan}, {\bf 55}, 1--11.

\bibitem{Masuda8}
Masuda, T. (2003). On non-strongly free automorphisms of subfactors of type III$_0$.
{\em Canadian Mathematical Bulletin}, {\bf 46}, 419--428.

\bibitem{Masuda9}
Masuda, T. (2005). An analogue of Connes-Haagerup approach to classification of
subfactors of type $III_1$. {\em Journal of the Mathematical Society of Japan}, {\bf 57}, 959--1001.

\bibitem{MWu}
McCoy, B. and Wu, T. (1972). The two dimensional Ising model.
{\em Harvard University Press, Cambridge, Massachusetts}, {\bf 40}.

\bibitem{McD1}
McDuff, D. (1969). Uncountably many $II_1$ factors. {\em Annals of Mathematics}, {\bf 90}, 372--377.

\bibitem{McD2}
McDuff, D. (1970). Central sequences and the hyperfinite factor.
{\em Proceedings of the London Mathematical Society},
{\bf 21}, 443--461.

\bibitem{MS}
Moore, G. and Seiberg, N. (1989). Classical and quantum conformal field theory.
{\em Communications in Mathematical Physics},
{\bf 123}, 177--254.

\bibitem{MS1}
Moore, G. and Seiberg, N. (1989). Naturality in conformal field theory.
{\em Nuclear Physics B}, {\bf 313}, 16--40.

\bibitem{Mg1}
M\"uger, M. (1998). Superselection structure of massive quantum field theories in $1+1$
dimensions. {\em Reviews in Mathematical Physics}, {\bf 10}, 1147--1170.

\bibitem{Mg2}
M\"uger, M. (1999). On soliton automorphisms in massive and conformal theories.
{\em Reviews in Mathematical Physics}, {\bf 11}, 337--359.

\bibitem{Mg3}
M\"uger, M. (1999). On charged fields with group symmetry and degeneracies of Verlinde's
matrix $S$. {\em Annales de l'Institut Henri Poincar\'e. Physique Th\'eorique},
{\bf 71}, 359--394.

\bibitem{Mg4}
M\"uger, M. (2000). Galois theory for braided tensor categories and the modular closure.
{\em Advances in Mathematics}, {\bf 150}, 151--201.

\bibitem{Mg5}
M\"uger, M. (2001). Conformal field theory and Doplicher-Roberts reconstruction.
in {\em Mathematical Physics in Mathematics and Physics} (ed. R. Longo),
The Fields Institute Communications {\bf 30}, Providence, Rhode Island:
AMS Publications, 297--319. math-ph/0008027.

\bibitem{Mg6}
M\"uger, M. (2003). From subfactors to categories and topology I.
Frobenius algebras in and Morita equivalence of tensor categories.
{\em Journal of Pure and Applied Algebra}, {\bf 180}, 81--157. math.CT/0111204.

\bibitem{Mg7}
M\"uger, M. (2003). From subfactors to categories and topology II.
The quantum double of subfactors and categories.
{\em Journal of Pure and Applied Algebra}, {\bf 180}, 159--219. math.CT/0111205.

\bibitem{Mg8}
M\"uger, M. (in press).
On the structure of modular categories. {\em Proceedings of the London Mathematical Society}. math.CT/0201017.

\bibitem{Mg9}
M\"uger, M. (preprint 2002). Galois extensions of braided tensor categories and
braided crossed G-categories. math.CT/0209093.

\bibitem{Mg10}
M\"uger, M. (2005). Conformal orbifold theories and braided crossed G-categories
{\em Communications in Mathematical Physics}, {\bf 260}, 727--762. math.QA/0403322.

\bibitem{MW}
Munemasa, A. and Watatani, Y. (1992). Paires orthogonales de sous-algebres involutives.
{\em Comptes Rendus de l'Academie des Sciences,
S\'erie I, Math\'ematiques}, {\bf 314}, 329--331.

\bibitem{MuJ}
Murakami, J. (1987). The Kauffman polynomial of lins and representation theory.
{\em Osaka Journal of Mathematics}, {\bf 24}, 745--758.

\bibitem{MuH}
Murakami, H. (1994). Quantum $SU(2)$-invariants dominate Casson's
$SU(2)$-invariant. {\em Mathematical Proceedings of the Cambridge Philosophical Society},
{\bf 115}, 253--281.

\bibitem{Mur}
Murphy, G. J. (1990). $C^*$-Algebras and Operator Theory. {\em Academic Press Incorporated}.

\bibitem{Murray}
Murray, F. J. (1990). The rings of operators. {\em Symposia Mathematica},{\bf 50}, 57--60.

\bibitem{MvN1}
Murray, F. J. and von Neumann, J. (1936). On rings of operators.
{\em Annals of Mathematics},
{\bf 37}, 116--229.

\bibitem{MvN2}
Murray, F. J. and von Neumann, J. (1937). On rings of operators II.
{\em Transactions of the American Mathematical Society},
{\bf 41}, 208--248.

\bibitem{MvN3}
Murray, F. J. and von Neumann, J. (1943). On rings of operators IV.
{\em Annals of Mathematics}, {\bf 44}, 716--808.

\bibitem{Na}
Nahm, W. (1988). Lie group exponents and $SU(2)$ current algebras.
{\em Communications in Mathematical Physics}, {\bf 118}, 171--176.

\bibitem{Nahm}
Nahm, W. (1991). A proof of modular invariance. {\em International Journal of Modern Physics},
{\bf 6}, 2837--2845.

\bibitem{NT}
Nakanishi, T. and Tsuchiya, A. (1992). Level-rank duality of $WZW$ models in conformal field theory.
{\em Communications in Mathematical Physics}, {\bf 144}, 351--372.

\bibitem{vN1}
von Neumann, J. (1935). Charakterisierung des Spektrums eines Integral Operators.
{\em Act. Sci. et Ind.}, {\bf 229}.

\bibitem{vN2} von Neumann, J. (1940). On Rings of Operators III. {\em Annals of Mathematics}, {\bf 41}, 94--161.

\bibitem{Neu}
von Neumann, J. (1961). Collected works vol. III. Rings of Operators. Pergamon Press.

\bibitem{NV}
Nikshych, D. and Vainerman, L. (2000).
A Galois correspondence for $II_1$ factors and quantum groupoids.
{\em Journal of Functional Analysis}, {\bf 178}, 113--142.

\bibitem{NW}
Nill, F. and Wiesbrock, H.-W. (1995). A comment on Jones inclusions with infinite index. {\em Reviews in Mathematical Physics}, {\bf 7}, 599--630.

\bibitem{O1}
Ocneanu, A. (1985). {\em Actions of discrete amenable groups on factors},
Lecture Notes in Mathematics {\bf 1138}, Springer, Berlin.

\bibitem{O2}
Ocneanu, A. (1988). Quantized group, string algebras and Galois theory for algebras.
{\em Operator algebras and applications, Vol. 2 (Warwick, 1987)},
(ed. D. E. Evans and M. Takesaki), London Mathematical Society
Lecture Note Series Vol. 136, Cambridge University Press, 119--172.

\bibitem{O3}
Ocneanu, A. (1991). {\em Quantum symmetry, differential geometry of
finite graphs and classification of subfactors},
University of Tokyo Seminary Notes 45, (Notes recorded by Kawahigashi, Y.).

\bibitem{O4}
Ocneanu, A. (1994). Chirality for operator algebras. (recorded by Kawahigashi, Y.)
in {\em Subfactors --- Proceedings of the Taniguchi Symposium, Katata ---},
(ed. H. Araki, et al.), World Scientific, 39--63.

\bibitem{O5}
Ocneanu, A. (2000).
Paths on Coxeter diagrams: from Platonic solids and singularities
to minimal models and subfactors. (Notes recorded by S. Goto), in
{\em Lectures on operator theory}, (ed. B. V. Rajarama Bhat et al.), The Fields Institute Monographs, Providence, Rhode Island: AMS Publications, 243--323.

\bibitem{O6}
Ocneanu, A. (2001). Operator algebras, topology and subgroups of quantum symmetry
--construction of subgroups of quantum groups--. (recorded by Goto, S. and Sato, N.) in {\em Taniguchi Conference on Mathematics Nara'98}, Advanced Studies in Pure Mathematics {\bf 31}, (ed. M. Maruyama and T. Sunada), Mathematical Society of Japan, 235--263.

\bibitem{O7}
Ocneanu, A. (2002).
The classification of subgroups of quantum $SU(N)$.
in {\em Quantum Symmetries in Theoretical Physics and
Mathematics} (ed. R. Coquereaux et al.), Comtemp. Math. {\bf 294}, Amer. Math. Soc., 133--159.

\bibitem{Ok}
Okamoto, S. (1991). Invariants for subfactors arising from Coxeter
graphs. {\em Current Topics in Operator Algebras},
World Scientific Publishing, 84--103.

\bibitem{On1}
Onsager, L. (1941). Crystal Statistics I. {\em Physical Review}, {\bf 65}, 117--149.

\bibitem{On2}
Onsager, L. (1949). Discussion remark (Spontaneous magnetisation of the
two-dimensional Ising model. {\em Nuovo Cimento, Supplement}, {\bf 6}, 261--262.

\bibitem{Os}
Ostrik, V. (preprint 2001). Module categories, weak Hopf algebras and modular invariants.
math.RT/0111140.

\bibitem{Os2}
Ostrik, V. (2003).
Module categories over the Drinfeld double of a finite group.
{\em International Mathematics Research Notices}, 1507-1520.

\bibitem{Oz}
Ozawa, N. (in press). No separable $II_1$-factor can contain all separable $II_1$-factors
as its subfactors. {\em Proceedings of the American Mathematical Society}. math.OA/0210411

\bibitem{Maltsiniotis}
G. Maltsiniotis.: Groupo$\ddot{i}$des quantiques., C. R. Acad. Sci. Paris, \textbf{314}:
249 -- 252.(1992)

\bibitem{May}
J.P. May: A Concise Course in Algebraic Topology. Chicago and London: The Chicago University
Press. (1999).

\bibitem{Mitchell}
B. Mitchell: \emph{The Theory of Categories}, Academic Press,
London, (1965).

\bibitem{Mitchell72}
B. Mitchell: Rings with several objects., \textit{Adv. Math}. \textbf{8}: 1 - 161 (1972).

\bibitem{Mitchell85}
B. Mitchell: Separable Algebroids., {\em M. American Math. Soc.} \textbf{333}:10-19 (1985).

\bibitem{Mo}
G. H. Mosa: \emph{Higher dimensional algebroids and Crossed
complexes}, PhD thesis, University of Wales, Bangor, (1986).

\bibitem{Moultaka}
G. Moultaka, M. Rausch de Traubenberg and A. Tanas\u a: Cubic
supersymmetry and abelain gauge invariance, \emph{Internat. J.
Modern Phys. A} \textbf{20} no. 25 (2005), 5779--5806.

\bibitem{Mrcun}
J. Mr\v cun : On spectral representation of coalgebras and Hopf
algebroids, (2002) preprint.\\ http://arxiv.org/abs/math/0208199.

\bibitem{MRW}
P. Muhli, J. Renault and D. Williams: Equivalence and isomorphism for groupoid C*--algebras., \textit{J. Operator Theory}, \textbf{17}: 3-22 (1987).

\bibitem{Nikshych}
D. A. Nikshych and L. Vainerman: \emph{J. Funct. Anal.}
\textbf{171} (2000) No. 2, 278--307

\bibitem{Nishimura96}
H. Nishimura.: Logical quantization of topos theory., \textit{International Journal of Theoretical Physics}, Vol. \textbf{35},(No. 12): 2555--2596 (1996).

\bibitem{NM}
M. Neuchl, PhD Thesis, University of Munich (1997).

\bibitem{OV06}
V. Ostrik.: Module Categories over Representations of $SL_q(2)$
in the Non--simple Case. (2006).
\\ http://arxiv.org/PS--cache/math/pdf/0509/0509530v2.pdf

\bibitem{PAk3}
A. L. T. Paterson: The Fourier-Stieltjes and Fourier algebras for locally
compact groupoids., \textit{Contemporary Mathematics} \textbf{321}: 223-237 (2003)

\bibitem{PLR}
R. J. Plymen and P. L. Robinson : {\em Spinors in Hilbert Space}.
\emph{Cambridge Tracts in Math.} \textbf{114}, \emph{Cambridge Univ. Press} (1994).

\bibitem{NPopescu}
N. Popescu.: \emph{The Theory of Abelian Categories with Applications to Rings and
Modules}, New York and London: Academic Press (1968).

\bibitem{Prigogine}
I. Prigogine: \textit{From being to becoming: time and complexity in the physical sciences}. W. H. Freeman and Co: San Francisco (1980).

\bibitem{RW1}
A. Ramsay: Topologies on measured groupoids., \textit{J. Functional Analysis}, \textbf{47}: 314-343 (1982).

\bibitem{RW2}
A. Ramsay and M. E. Walter: Fourier-Stieltjes algebras of locally compact groupoids.,
\textit{J. Functional Analysis}, \textbf{148}: 314-367 (1997).

\bibitem{RZ}
I. Raptis and R. R. Zapatrin : Quantisation of discretized
spacetimes and the correspondence principle, \emph{Int. Jour.Theor. Phys.} \textbf{39},(1), (2000).

\bibitem{REG}
T. Regge.: General relativity without coordinates. \textit{Nuovo Cimento} (\textbf{10}) 19: 558\^a~@~S571 (1961).

\bibitem{Rehren}
H.--K. Rehren: Weak C*--Hopf symmetry, \emph{Quantum Group
Symposium at Group 21, Proceedings, Goslar (1996}, Heron Ptess, Sofia BG : 62--69(1997).

\bibitem{Renault1}
J. Renault: A groupoid approach to C*-algebras. Lecture Notes in Maths. \textbf{793}, Berlin: Springer-Verlag,(1980).

\bibitem{Renault2}
J. Renault: Representations de produits croises d'algebres de groupoides. ,
\textit{J. Operator Theory}, \textbf{18}:67-97 (1987).

\bibitem{Renault3}
J. Renault: The Fourier algebra of a measured groupoid and its multipliers.,
\textit{J. Functional Analysis}, \textbf{145}: 455-490 (1997).

\bibitem{Rieffel1}
M. A. Rieffel: Group C*--algebras as compact quantum metric spaces, \emph{Documenta Math.} \textbf{7}: 605-651 (2002).

\bibitem{Rieffel2}
M. A. Rieffel: Induced representations of C*-algebras, \emph{Adv. in Math.} \textbf{13}: 176-254 (1974).

\bibitem{noncomm}
J. E. Roberts : More lectures on algebraic quantum field theory (in A. Connes, et al., {\em Noncommutative Geometry}), Springer: Berlin (2004).

\bibitem{RJ2}
J. Roberts.: Skein theory and Turaev-Viro invariants. \textit{Topology} \textbf{34}(no.4):771-787(1995).

\bibitem{RJ3} J. Roberts. Refined state-sum invariants of 3- and 4- manifolds. Geometric topology
(Athens, GA, 1993), 217-234, \textit{AMS/IP Stud. Adv. Math}., \textbf{2.1}, Amer. Math.Soc., Providence, RI,(1997).

\bibitem{RC98}
Rovelli, C.: 1998, Loop Quantum Gravity, in N. Dadhich, et al. {\em Living Reviews in Relativity}.


\bibitem{Schwartz45}
Schwartz, L.: Generalisation de la Notion de Fonction, de Derivation, de Transformation
de Fourier et Applications Mathematiques et Physiques, \textit{Annales de l'Universite de Grenoble},
21: 57-74 (1945).

\bibitem{Schwartz52}
Schwartz, L.: Th\'eorie des Distributions, Publications de l'Institut de Math\'ematique
de l'Universit\'e de Strasbourg, Vols 9-10, Paris: Hermann (1951-1952).

\bibitem{Seda1}
A. K. Seda: Haar measures for groupoids, \emph{Proc. Roy. Irish Acad.
Sect. A} \textbf{76} No. 5, 25--36 (1976).

\bibitem{Seda2}
A. K. Seda: Banach bundles of continuous functions and an integral
representation theorem, \emph{Trans. Amer. Math. Soc.} \textbf{270} No.1 : 327-332(1982).

\bibitem{Seda1}
A. K. Seda: On the Continuity of Haar measures on topological groupoids, \emph{Proc. Amer Mat. Soc.} \textbf{96}: 115--120 (1986).

\bibitem{Segal47b}
I. E. Segal, Postulates for General Quantum Mechanics, {\em Annals of Mathematics}, 4:
930-948 (1947b).

\bibitem{Segal47a}
I.E. Segal.: Irreducible Representations of Operator Algebras, {\em Bulletin of the
American Mathematical Society}, 53: 73-88 (1947a).

\bibitem{Sklyanin83}
E.K. Sklyanin: Some Algebraic Structures Connected with the Yang-Baxter equation, Funct. Anal. Appl., \textbf{16}: 263--270 (1983).

\bibitem{Sklyanin84}
Some Algebraic Structures Connected with the Yang-Baxter equation. Representations of Quantum Algebras, Funct. Anal.Appl., 17: 273-284 (1984).

\bibitem{Szlach}
K. Szlach\'anyi: The double algebraic view of finite quantum
groupoids, \emph{J. Algebra} \textbf{280} (1) , 249-294 (2004).

\bibitem{Sweedler96}
M. E. Sweedler: \textit{Hopf algebras.} W.A. Benjamin, INC., New
York (1996).

\bibitem{Tanasa}
A. Tanas\u a: Extension of the Poincar\'e symmetry and its field
theoretical interpretation, \emph{SIGMA Symmetry Integrability
Geom. Methods Appl.} \textbf{2} (2006), Paper 056, 23 pp.
(electronic).

\bibitem{taylorj:88}
J.~{T}aylor, {\em Quotients of Groupoids by the Action of a
Group\/}, Math. Proc. {C}amb. Phil. Soc., 103, (1988), 239--249.

\bibitem{tonksthesis}
A.~P. Tonks, 1993, {\em Theory and applications of crossed
complexes\/}, Ph.D. thesis, University of Wales, Bangor.

\bibitem{TV}
V.G. Turaev and O.Ya. Viro. State sum invariants of 3-manifolds and quantum
6j-symbols. \textit{Topology} \textbf{31} (no. 4): 865-902 (1992).

\bibitem{Varilly}
J. C. V\'arilly: An introduction to noncommutative geometry, (1997)
arXiv:physics/9709045

\bibitem{van kampen1}
E.~H.~van {Kampen}, {\em On the Connection Between the Fundamental
Groups of some Related Spaces\/}, Amer. J. Math., 55, (1933), 261--267.

\bibitem{Xu}
P. Xu.: Quantum groupoids and deformation quantization. (1997)
(at arxiv.org/pdf/q--alg/9708020.pdf).

\bibitem{Y}
D.N. Yetter.: TQFT's from homotopy 2-types. \textit{J. Knot Theor}. \textbf{2}:113-123(1993).

\bibitem{Weinberg}
S. Weinberg.: \emph{The Quantum Theory of Fields}. Cambridge, New York and Madrid: %%@
Cambridge University Press, Vols. 1 to 3, (1995--2000).

\bibitem{Weinstein}
A. Weinstein : Groupoids: unifying internal and external symmetry,
\emph{Notices of the Amer. Math. Soc.} \textbf{43} (7): 744--752 (1996).

\bibitem{WB}
J. Wess and J. Bagger: \emph{Supersymmetry and Supergravity},
Princeton University Press, (1983).

\bibitem{WJ1}
J. Westman: Harmonic analysis on groupoids, \textit{Pacific J. Math.} \textbf{27}: 621-632. (1968).

\bibitem{WJ1}
J. Westman: Groupoid theory in algebra, topology and analysis., \textit{University of California at Irvine} (1971).

\bibitem{Wickra}
S. Wickramasekara and A. Bohm: Symmetry representations in the rigged Hilbert space formulation of quantum mechanics, \emph{J. Phys. A} \textbf{35} (2002), no. 3, 807-829.

\bibitem{Wightman1}
Wightman, A. S., 1956, Quantum Field Theory in Terms of Vacuum Expectation Values, \emph{Physical Review}, \textbf{101}: 860--866.

\bibitem{Wightman2}
Wightman, A. S., 1976, Hilbert's Sixth Problem: Mathematical Treatment of the Axioms of Physics, \emph{Proceedings of Symposia in Pure Mathematics}, \textbf{28}: 147--240.

\bibitem{Wightman--Garding3}
Wightman, A.S. and G\aa{}rding, L., 1964, Fields as Operator--Valued Distributions in
Relativistic Quantum Theory, \emph{Arkiv fur Fysik}, 28: 129--184.

\bibitem{Woronowicz1}
S. L. Woronowicz : Twisted $SU(2)$ group : An example of a non-commutative differential calculus, RIMS, Kyoto University \textbf{23} 1987), 613-665.


\end{thebibliography}

%%%%%
%%%%%
\end{document}
