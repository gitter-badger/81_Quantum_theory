\documentclass[12pt]{article}
\usepackage{pmmeta}
\pmcanonicalname{Quantization}
\pmcreated{2013-03-22 15:36:59}
\pmmodified{2013-03-22 15:36:59}
\pmowner{bci1}{20947}
\pmmodifier{bci1}{20947}
\pmtitle{quantization}
\pmrecord{27}{37540}
\pmprivacy{1}
\pmauthor{bci1}{20947}
\pmtype{Topic}
\pmcomment{trigger rebuild}
\pmclassification{msc}{81S10}
\pmclassification{msc}{53D50}
\pmclassification{msc}{46L65}
\pmclassification{msc}{81R50}
\pmclassification{msc}{81R15}
\pmsynonym{quantisation}{Quantization}
\pmsynonym{canonical quantization}{Quantization}
\pmsynonym{Weyl-quantization}{Quantization}
%\pmkeywords{quantization}
%\pmkeywords{symplectic}
%\pmkeywords{Poisson}
%\pmkeywords{Hamiltonian operator}
%\pmkeywords{Lie algebra}
%\pmkeywords{Clifford algebra}
\pmrelated{CCliffordAlgebra}
\pmrelated{AsymptoticMorphismsAndWignerWeylMoyalQuantizationProcedures}
\pmrelated{HamiltonianEquations}
\pmrelated{PoissonBracket}
\pmrelated{SchrodingersWaveEquation}
\pmrelated{CanonicalQuantization}
\pmrelated{HamiltonianOperatorOfAQuantumSystem}
\pmrelated{QuantumSpaceTimes}
\pmrelated{QFTOrQuantumFieldTheories}
\pmrelated{QEDInTh}
\pmdefines{classical system}
\pmdefines{classical state}
\pmdefines{classical observable}
\pmdefines{quantum system}
\pmdefines{quantum state}
\pmdefines{quantum observable}

\usepackage{amssymb, amsmath, amsthm}
%%%\usepackage{xypic}

\newtheorem{thm}{Theorem}
\newtheorem{prop}[thm]{Proposition} 
\newtheorem{lemma}[thm]{Lemma}
\newtheorem{cor}[thm]{Corollary}

\theoremstyle{definition} 
\newtheorem{dfn}[thm]{Definition}

\theoremstyle{remark}
\newtheorem*{rmk}{Remark}
\newtheorem{ex}{Example}

\newcommand{\lie}{\mathcal{L}}
\newcommand{\pdiff}[2]{\frac{\partial #1}{\partial #2}}
\newcommand{\grad}{\nabla}
\begin{document}
\subsection{Introduction}
Quantization is understood as the process of defining a \emph{formal} correspondence between a quantum system operator (such as the quantum Hamiltonian operator) or quantum algebra and a classical system operator (such as the Hamiltonian) or a classical algebra, such as the Poisson algebra.  Theoretical quantum physicists often proceed in two `stages', so that both first and second quantization procedures were reported in QFT, for example. Generalized quantization procedures involve \emph{asymptotic morphisms and Wigner--Weyl--Moyal quantization procedures} or noncommutative `deformations' of \PMlinkname{C*-algebras}{CAlgebra3} associated with quantum operators on Hilbert spaces (as in noncommutative geometry). The \emph{non-commutative} algebra of quantum observable operators is a 
\PMlinkname{Clifford algebra}{CliffordAlgebra}, and the associated 
\PMlinkname{$C^*$-Clifford algebra}{CCliffordAlgebra} is a fundamental concept of modern mathematical treatments of quantum theories. Note, however, that classical systems, including Einstein's general relativity are commutative (or Abelian) theories, whereas quantum theories are intrinsically non-commutative (or non-Abelian), most likely as a consequnece of the non-comutativity of quantum logics and the Heisenberg uncertainty principle of quantum mechanics. 

This definition is quite broad, and as a result there are many approaches to quantization, employing a variety of techniques.  It should be emphasized the result of quantization is not unique; in fact, methods of quantization usually possess inherent ambiguities, in the sense that, while performing quantization, one usually must make choices at certain points of the process.

\subsection*{Classical systems}

\begin{dfn}A \emph{classical system} is a triplet $(M, \omega, H)$, where $(M, \omega)$ (the \emph{phase space}) is a symplectic manifold and $H$ (the \emph{Hamiltonian}) is a smooth function on $M$.
\end{dfn}

In most physical examples the phase space $M$ is the cotangent bundle $T^*X$ of a manifold $X$.  In this case, $X$ is called the \emph{configuration space}.  

\begin{dfn}
\begin{enumerate}
\item A \emph{classical state} is a point $x$ in $M$.
\item A \emph{classical observable} is a function $f$ on $M$.
\end{enumerate}
\end{dfn}

In classical mechanics, one studies the time-evolution of a classical system.  The time-evolution of an observable is described the equation
\begin{equation}\label{ham}
\frac{df}{dt} = -\{H,f\},
\end{equation}
where $\{\cdot, \cdot\}$ is the Poisson bracket.  Equation (\ref{ham}) is equivalent to the Hamilton equations.

\begin{rmk}
A classical system is sometimes defined more generally as a triplet $(M, \pi, H)$, where $\pi$ is a Poisson structure on $M$.  
\end{rmk}

\subsection*{Quantum systems}

\begin{dfn}A \emph{quantum system} is a pair $(\mathcal{H}, \hat{H})$, where $\mathcal{H}$ is a Hilbert space and $\hat{H}$ is a self-adjoint linear operator on $\mathcal{H}$.
\end{dfn}

If $(\mathcal{H}, \hat{H})$ is a quantum system, $\mathcal{H}$ is referred to as the (quantum) phase space and $\hat{H}$ is referred to as the \emph{Hamiltonian operator}.

\begin{dfn}
\begin{enumerate}
\item A \emph{quantum state} is a vector $\Psi$ in $\mathcal{H}$.
\item A \emph{quantum observable} is a self-adjoint linear operator $A$ on $\mathcal{H}$.
\end{enumerate}
\end{dfn}

The space of quantum observables is denoted $\mathcal{O}(\mathcal{H})$.  If $A$ and $B$ are in $\mathcal{O}(\mathcal{H})$, then 
\begin{equation}\label{salie}
(i\hbar)^{-1}[A,B] := (i\hbar)^{-1}(AB - BA)
\end{equation}
is in $\mathcal{O}(\mathcal{H})$ (Planck's constant $\hbar$ appears as a scaling factor arising from physical considerations).  The operation $(i\hbar)^{-1}[\cdot,\cdot]$ thus gives $\mathcal{O}(\mathcal{H})$ the structure of a Lie algebra.

The time evolution of a quantum observable is described by the equation
\begin{equation}\label{schro}
\frac{dA}{dt} =  \frac{i}{\hbar} [\hat{H}, A].
\end{equation}
Equation (\ref{schro}) is equivalent to the \emph{time-dependent} Schr\"{o}dinger's equation
\begin{equation}
i \hbar \frac{d\Psi}{dt} = \hat{H} \Psi.
\end{equation}

\subsection*{The problem of quantization}

The problem of quantization is to find a correspondence between a quantum system and a classical system; this is clearly not always possible. Thus, specific methods of quantization describe several ways of constructing a pair $(\mathcal{H}, \hat{H})$ from a triplet $(M, \omega, H)$.  Furthermore, in order to give physical meaning to the observables in the quantum system, there should be a map
\begin{equation}
q\colon C^\infty(M) \to \mathcal{O}(\mathcal{H}),
\end{equation}
satisfying the following conditions:
\begin{enumerate}
\item $q$ is a Lie algebra homomorphism,
\item $q(H) = \hat{H}$.
\end{enumerate}

\begin{rmk} Note that $q$ is not an algebra homomorphism.  Much of the complexity of quantization lies in the fact that, while $C^\infty(M)$ is a commutative algebra, its image in $\mathcal{O}(\mathcal{H})$ necessarily does not commute.
\end{rmk}

The following is a list of some well-known methods of quantization:
\begin{itemize}
\item Canonical quantization
\item Geometric quantization
\item Deformation quantization
\item Path-integral quantization
\end{itemize}

A detailed example of geometric quantization on quantum Riemannian spaces can be found in 
ref. \cite{AL2k5}.

\begin{thebibliography}{9}
\bibitem{AL2k5}
Abhay Ashtekar and Jerzy Lewandowski. 2005. Quantum Geometry and Its Applications.
\PMlinkexternal{Available PDF download}{http://cgpg.gravity.psu.edu/people/Ashtekar/articles/qgfinal.pdf}.
\end{thebibliography}

%%%%%
%%%%%
\end{document}
