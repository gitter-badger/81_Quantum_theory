\documentclass[12pt]{article}
\usepackage{pmmeta}
\pmcanonicalname{Ralgebroid}
\pmcreated{2013-03-22 18:14:19}
\pmmodified{2013-03-22 18:14:19}
\pmowner{bci1}{20947}
\pmmodifier{bci1}{20947}
\pmtitle{R-algebroid}
\pmrecord{25}{40827}
\pmprivacy{1}
\pmauthor{bci1}{20947}
\pmtype{Definition}
\pmcomment{trigger rebuild}
\pmclassification{msc}{81T10}
\pmclassification{msc}{81P05}
\pmclassification{msc}{81T05}
\pmclassification{msc}{81R10}
\pmclassification{msc}{81R50}
\pmsynonym{groupoid-derived algebroids}{Ralgebroid}
\pmsynonym{double groupoid dual of an algebroid}{Ralgebroid}
%\pmkeywords{defintions of R-algebroid}
%\pmkeywords{R-Category}
%\pmkeywords{Groupoid-derived algebroid}
\pmrelated{Module}
\pmrelated{RCategory}
\pmrelated{Algebroids}
\pmrelated{HamiltonianAlgebroids}
\pmrelated{RSupercategory}
\pmrelated{SuperalgebroidsAndHigherDimensionalAlgebroids}
\pmrelated{CategoricalAlgebras}
\pmdefines{$R$-module}
\pmdefines{convolution product}
\pmdefines{R-algebroid}

% this is the default PlanetMath preamble.  as your knowledge
% of TeX increases, you will probably want to edit this, but
% it should be fine as is for beginners.

% almost certainly you want these
\usepackage{amssymb}
\usepackage{amsmath}
\usepackage{amsfonts}
 
% define commands here
\usepackage{amsmath, amssymb, amsfonts, amsthm, amscd, latexsym}
%%\usepackage{xypic}
\usepackage[mathscr]{eucal}
\theoremstyle{plain}
\newtheorem{lemma}{Lemma}[section]
\newtheorem{proposition}{Proposition}[section]
\newtheorem{theorem}{Theorem}[section]
\newtheorem{corollary}{Corollary}[section]

\theoremstyle{definition}
\newtheorem{definition}{Definition}[section]
\newtheorem{example}{Example}[section]
%\theoremstyle{remark}
\newtheorem{remark}{Remark}[section]
\newtheorem*{notation}{Notation}
\newtheorem*{claim}{Claim}

\renewcommand{\thefootnote}{\ensuremath{\fnsymbol{footnote%%@
}}}
\numberwithin{equation}{section}

\newcommand{\Ad}{{\rm Ad}}
\newcommand{\Aut}{{\rm Aut}}
\newcommand{\Cl}{{\rm Cl}}
\newcommand{\Co}{{\rm Co}}
\newcommand{\DES}{{\rm DES}}
\newcommand{\Diff}{{\rm Diff}}
\newcommand{\Dom}{{\rm Dom}}
\newcommand{\Hol}{{\rm Hol}}
\newcommand{\Mon}{{\rm Mon}}
\newcommand{\Hom}{{\rm Hom}}
\newcommand{\Ker}{{\rm Ker}}
\newcommand{\Ind}{{\rm Ind}}
\newcommand{\IM}{{\rm Im}}
\newcommand{\Is}{{\rm Is}}
\newcommand{\ID}{{\rm id}}
\newcommand{\GL}{{\rm GL}}
\newcommand{\Iso}{{\rm Iso}}
\newcommand{\Sem}{{\rm Sem}}
\newcommand{\St}{{\rm St}}
\newcommand{\Sym}{{\rm Sym}}
\newcommand{\SU}{{\rm SU}}
\newcommand{\Tor}{{\rm Tor}}
\newcommand{\U}{{\rm U}}

\newcommand{\A}{\mathcal A}
\newcommand{\Ce}{\mathcal C}
\newcommand{\D}{\mathcal D}
\newcommand{\E}{\mathcal E}
\newcommand{\F}{\mathcal F}
\newcommand{\G}{\mathcal G}
\newcommand{\Q}{\mathcal Q}
\newcommand{\R}{\mathcal R}
\newcommand{\cS}{\mathcal S}
\newcommand{\cU}{\mathcal U}
\newcommand{\W}{\mathcal W}

\newcommand{\bA}{\mathbb{A}}
\newcommand{\bB}{\mathbb{B}}
\newcommand{\bC}{\mathbb{C}}
\newcommand{\bD}{\mathbb{D}}
\newcommand{\bE}{\mathbb{E}}
\newcommand{\bF}{\mathbb{F}}
\newcommand{\bG}{\mathbb{G}}
\newcommand{\bK}{\mathbb{K}}
\newcommand{\bM}{\mathbb{M}}
\newcommand{\bN}{\mathbb{N}}
\newcommand{\bO}{\mathbb{O}}
\newcommand{\bP}{\mathbb{P}}
\newcommand{\bR}{\mathbb{R}}
\newcommand{\bV}{\mathbb{V}}
\newcommand{\bZ}{\mathbb{Z}}

\newcommand{\bfE}{\mathbf{E}}
\newcommand{\bfX}{\mathbf{X}}
\newcommand{\bfY}{\mathbf{Y}}
\newcommand{\bfZ}{\mathbf{Z}}

\renewcommand{\O}{\Omega}
\renewcommand{\o}{\omega}
\newcommand{\vp}{\varphi}
\newcommand{\vep}{\varepsilon}

\newcommand{\diag}{{\rm diag}}
\newcommand{\grp}{{\mathbb G}}
\newcommand{\dgrp}{{\mathbb D}}
\newcommand{\desp}{{\mathbb D^{\rm{es}}}}
\newcommand{\Geod}{{\rm Geod}}
\newcommand{\geod}{{\rm geod}}
\newcommand{\hgr}{{\mathbb H}}
\newcommand{\mgr}{{\mathbb M}}
\newcommand{\ob}{{\rm Ob}}
\newcommand{\obg}{{\rm Ob(\mathbb G)}}
\newcommand{\obgp}{{\rm Ob(\mathbb G')}}
\newcommand{\obh}{{\rm Ob(\mathbb H)}}
\newcommand{\Osmooth}{{\Omega^{\infty}(X,*)}}
\newcommand{\ghomotop}{{\rho_2^{\square}}}
\newcommand{\gcalp}{{\mathbb G(\mathcal P)}}

\newcommand{\rf}{{R_{\mathcal F}}}
\newcommand{\glob}{{\rm glob}}
\newcommand{\loc}{{\rm loc}}
\newcommand{\TOP}{{\rm TOP}}

\newcommand{\wti}{\widetilde}
\newcommand{\what}{\widehat}

\renewcommand{\a}{\alpha}
\newcommand{\be}{\beta}
\newcommand{\ga}{\gamma}
\newcommand{\Ga}{\Gamma}
\newcommand{\de}{\delta}
\newcommand{\del}{\partial}
\newcommand{\ka}{\kappa}
\newcommand{\si}{\sigma}
\newcommand{\ta}{\tau}
\newcommand{\med}{\medbreak}
\newcommand{\medn}{\medbreak \noindent}
\newcommand{\bign}{\bigbreak \noindent}
\newcommand{\lra}{{\longrightarrow}}
\newcommand{\ra}{{\rightarrow}}
\newcommand{\rat}{{\rightarrowtail}}
\newcommand{\oset}[1]{\overset {#1}{\ra}}
\newcommand{\osetl}[1]{\overset {#1}{\lra}}
\newcommand{\hr}{{\hookrightarrow}}

\begin{document}
\begin{definition}
If $\mathsf{G}$ is a groupoid (for example, regarded as a category with all morphisms invertible) 
then we can construct an $R$-algebroid, $R\mathsf{G}$ as follows. Let us consider first a module over a ring $R$, also called a {\em $R$-module}, that is, a \PMlinkname{module}{Module} $M_R$ that takes its coefficients in a ring $R$. Then, the object set of $R\mathsf{G}$ is the same as that of $\mathsf{G}$ and $R\mathsf{G}(b,c)$ is the free $R$-module on the set $\mathsf{G}(b,c)$, with composition given by the usual bilinear rule, extending the composition of $\mathsf{G}$.
\end{definition}

\begin{definition}
Alternatively, one can define $\bar{R}\mathsf{G}(b,c)$ to be the set of functions $\mathsf{G}(b,c)\lra R$ with finite support, and then one defines the \emph{convolution product} as follows:
\end{definition}
\med
\begin{equation}
(f*g)(z)= \sum \{(fx)(gy)\mid z=x\circ y \} ~.
\end{equation}


\begin{remark} 
 As it is very well known, only the second construction is natural
for the topological case, when one needs to replace the general concept of `function' by
the topological-analytical concept of `continuous function with compact support' (or alternatively, with `locally
compact support') for all quantum field theory (QFT) extended symmetry sectors; in this case, one has that $R \cong \mathbb{C}$~. 
The point made here is that to carry out the usual construction and end up with only an algebra
rather than an algebroid, is a procedure analogous to replacing a
groupoid $\mathsf{G}$ by a semigroup $G'=G\cup \{0\}$ in which the
compositions not defined in $G$ are defined to be $0$ in $G'$. We
argue that this construction removes the main advantage of
groupoids, namely the presence of the {\em spatial component} given by the set of objects of the groupoid. 
 
 More generally, a \PMlinkname{R-category}{RCategory} is similarly defined as an extension to this R-algebroid
concept. 
\end{remark}
\begin{thebibliography}{9}

\bibitem{BMos86}
R. Brown and G. H. Mosa: Double algebroids and crossed modules of algebroids, University of Wales--Bangor, Maths Preprint, 1986.

\bibitem{Mo86}
G. H. Mosa: \emph{Higher dimensional algebroids and Crossed
complexes}, PhD thesis, University of Wales, Bangor, (1986). (supervised by R. Brown).

\end{thebibliography}

%%%%%
%%%%%
\end{document}
