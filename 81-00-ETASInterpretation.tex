\documentclass[12pt]{article}
\usepackage{pmmeta}
\pmcanonicalname{ETASInterpretation}
\pmcreated{2013-03-22 18:16:04}
\pmmodified{2013-03-22 18:16:04}
\pmowner{bci1}{20947}
\pmmodifier{bci1}{20947}
\pmtitle{ETAS interpretation}
\pmrecord{80}{40870}
\pmprivacy{1}
\pmauthor{bci1}{20947}
\pmtype{Topic}
\pmcomment{trigger rebuild}
\pmclassification{msc}{81-00}
\pmclassification{msc}{92B05}
\pmclassification{msc}{03G30}
\pmclassification{msc}{18-00}
\pmsynonym{elementary theory of abstract supercategories}{ETASInterpretation}
\pmsynonym{ETAS}{ETASInterpretation}
%\pmkeywords{elementary theory of abstract supercategories}
%\pmkeywords{axiomatic theory of abstract supercategories}
\pmrelated{Category}
\pmrelated{CategoryTheory}
\pmrelated{ETAS}
\pmrelated{CategoricalOntologyABibliographyOfCategoryTheory}
\pmrelated{AlgebraicComputation}
\pmrelated{CategoricalOntology}
\pmrelated{QuantumLogic}
\pmrelated{CategoryOfQuantumAutomata}
\pmrelated{FunctorCategory2}
\pmrelated{QuantumAutomataAndQuantumComputation2}
\pmrelated{SupercategoryOfVariableMolecularSets}
\pmrelated{ETA}
\pmdefines{axioms of metacategories and supercategories}
\pmdefines{examples of supercategories and metacategories}
\pmdefines{ETAS interpretation}
\pmdefines{ETAS axiom}
\pmdefines{ETAS}

\endmetadata

% this is the default PlanetMath preamble.  as your 

% almost certainly you want these
\usepackage{amssymb}
\usepackage{amsmath}
\usepackage{amsfonts}

% define commands here
\usepackage{amsmath, amssymb, amsfonts, amsthm, amscd, latexsym}
%%\usepackage{xypic}
\usepackage[mathscr]{eucal}
\theoremstyle{plain}
\newtheorem{lemma}{Lemma}[section]
\newtheorem{proposition}{Proposition}[section]
\newtheorem{theorem}{Theorem}[section]
\newtheorem{corollary}{Corollary}[section]
\theoremstyle{definition}
\newtheorem{definition}{Definition}[section]
\newtheorem{example}{Example}[section]
%\theoremstyle{remark}
\newtheorem{remark}{Remark}[section]
\newtheorem*{notation}{Notation}
\newtheorem*{claim}{Claim}
\renewcommand{\thefootnote}{\ensuremath{\fnsymbol{footnote%%@
}}}
\numberwithin{equation}{section}
\newcommand{\Ad}{{\rm Ad}}
\newcommand{\Aut}{{\rm Aut}}
\newcommand{\Cl}{{\rm Cl}}
\newcommand{\Co}{{\rm Co}}
\newcommand{\DES}{{\rm DES}}
\newcommand{\Diff}{{\rm Diff}}
\newcommand{\Dom}{{\rm Dom}}
\newcommand{\Hol}{{\rm Hol}}
\newcommand{\Mon}{{\rm Mon}}
\newcommand{\Hom}{{\rm Hom}}
\newcommand{\Ker}{{\rm Ker}}
\newcommand{\Ind}{{\rm Ind}}
\newcommand{\IM}{{\rm Im}}
\newcommand{\Is}{{\rm Is}}
\newcommand{\ID}{{\rm id}}
\newcommand{\GL}{{\rm GL}}
\newcommand{\Iso}{{\rm Iso}}
\newcommand{\Sem}{{\rm Sem}}
\newcommand{\St}{{\rm St}}
\newcommand{\Sym}{{\rm Sym}}
\newcommand{\SU}{{\rm SU}}
\newcommand{\Tor}{{\rm Tor}}
\newcommand{\U}{{\rm U}}
\newcommand{\A}{\mathcal A}
\newcommand{\Ce}{\mathcal C}
\newcommand{\D}{\mathcal D}
\newcommand{\E}{\mathcal E}
\newcommand{\F}{\mathcal F}
\newcommand{\G}{\mathcal G}
\newcommand{\Q}{\mathcal Q}
\newcommand{\R}{\mathcal R}
\newcommand{\cS}{\mathcal S}
\newcommand{\cU}{\mathcal U}
\newcommand{\W}{\mathcal W}
\newcommand{\bA}{\mathbb{A}}
\newcommand{\bB}{\mathbb{B}}
\newcommand{\bC}{\mathbb{C}}
\newcommand{\bD}{\mathbb{D}}
\newcommand{\bE}{\mathbb{E}}
\newcommand{\bF}{\mathbb{F}}
\newcommand{\bG}{\mathbb{G}}
\newcommand{\bK}{\mathbb{K}}
\newcommand{\bM}{\mathbb{M}}
\newcommand{\bN}{\mathbb{N}}
\newcommand{\bO}{\mathbb{O}}
\newcommand{\bP}{\mathbb{P}}
\newcommand{\bR}{\mathbb{R}}
\newcommand{\bS}{\mathbb{S}}
\newcommand{\bV}{\mathbb{V}}
\newcommand{\bZ}{\mathbb{Z}}
\newcommand{\bfE}{\mathbf{E}}
\newcommand{\bfX}{\mathbf{X}}
\newcommand{\bfY}{\mathbf{Y}}
\newcommand{\bfZ}{\mathbf{Z}}
\renewcommand{\O}{\Omega}
\renewcommand{\o}{\omega}
\newcommand{\vp}{\varphi}
\newcommand{\vep}{\varepsilon}

\newcommand{\diag}{{\rm diag}}
\newcommand{\grp}{{\mathbb G}}
\newcommand{\dgrp}{{\mathbb D}}
\newcommand{\desp}{{\mathbb D^{\rm{es}}}}
\newcommand{\Geod}{{\rm Geod}}
\newcommand{\geod}{{\rm geod}}
\newcommand{\hgr}{{\mathbb H}}
\newcommand{\mgr}{{\mathbb M}}
\newcommand{\ob}{{\rm Ob}}
\newcommand{\obg}{{\rm Ob(\mathbb G)}}
\newcommand{\obgp}{{\rm Ob(\mathbb G')}}
\newcommand{\obh}{{\rm Ob(\mathbb H)}}
\newcommand{\Osmooth}{{\Omega^{\infty}(X,*)}}
\newcommand{\ghomotop}{{\rho_2^{\square}}}
\newcommand{\gcalp}{{\mathbb G(\mathcal P)}}
\newcommand{\rf}{{R_{\mathcal F}}}
\newcommand{\glob}{{\rm glob}}
\newcommand{\loc}{{\rm loc}}
\newcommand{\TOP}{{\rm TOP}}
\newcommand{\wti}{\widetilde}
\newcommand{\what}{\widehat}
\renewcommand{\a}{\alpha}
\newcommand{\be}{\beta}
\newcommand{\ga}{\gamma}
\newcommand{\Ga}{\Gamma}
\newcommand{\de}{\delta}
\newcommand{\del}{\partial}
\newcommand{\ka}{\kappa}
\newcommand{\si}{\sigma}
\newcommand{\ta}{\tau}
\newcommand{\lra}{{\longrightarrow}}
\newcommand{\ra}{{\rightarrow}}
\newcommand{\rat}{{\rightarrowtail}}
\newcommand{\oset}[1]{\overset {#1}{\ra}}
\newcommand{\osetl}[1]{\overset {#1}{\lra}}
\newcommand{\hr}{{\hookrightarrow}}
\begin{document}
\subsection{Introduction}
 
  \emph{ETAS} is the acronym for the {\em ``Elementary Theory of Abstract Supercategories''}
as defined by the {\em axioms of metacategories and supercategories}.


  The following are simple examples of supercategories that are essentially interpretations
of the eight {\em ETAC} axioms reported by W. F. Lawvere (1968), with one or several \emph{ETAS} axioms added as indicated in the examples listed. A family, or class, of a specific level (or 'order') $(n+1)$ of a
supercategory $\mathbb{S}_{n+1}$ (with $n$ being an integer) is defined by the specific ETAS axioms added to the eight ETAC axioms; thus, for $n=0$, there are no additional ETAS axioms and the supercategory $\mathbb{S}_1$ is the limiting, lower type, currently defined as a category with only one composition law and any standard interpretation of the eight ETAC axioms. Thus, the first level of 'proper' supercategory  $\mathbb{S}_2$ is defined as an interpretation of ETAS axioms \textbf{S1} and \textbf{S2}; for $n=3$, the supercategory $\mathbb{S}_4$ is defined as an interpretation of the eight ETAC axioms plus the additional three ETAS axioms: \textbf{S2}, \textbf{S3} and \textbf{S4}. Any (proper) recursive formula or 'function' can be utilized to generate supercategories at levels $n$ higher than $\mathbb{S}_4$ by adding composition or consistency laws to the ETAS axioms \textbf{S1} to \textbf{S4}, thus allowing a digital computer algorithm to generate any finite level supercategory $\mathbb{S}_n$ syntax, to which one needs then to add semantic interpretations (which are complementary to the computer generated syntax). 
 

\subsection{Simple examples of ETAS interpretation in supercategories}

\begin{enumerate}

\item {\em Functor categories} subject only to the eight {\em ETAC} axioms;
\item {\em Functor supercategories}, $\mathsf{\F_S}: \mathcal{A} \to \mathcal{B}$,
      with both $\mathcal{A}$ and $\mathcal{B}$ being 'large' categories (i.e., 
      $\mathcal{A}$ does not need to be small as in the case of {\em functor categories});
\item A \emph{topological groupoid category} is an example of a particular supercategory
      with all invertible morphisms endowed with both a topological and an agebraic
      structure, still subject to all ETAC axioms;
\item \emph{Supergroupoids} (also definable as crossed complexes of groupoids), and \emph{supergroups} --also definable as crossed modules of groups-- seem to be of great interest to mathematicians currently involved in `categorified' mathematical physics or physical mathematics.) 
\item A \emph{double groupoid category} is a `simple' example of a higher dimensional supercategory which is useful in higher dimensional homotopy theory, especially in non-Abelian algebraic topology;
this concept is subject to all eight ETAC axioms, plus additional axioms related to the definition of the double groupoid (generally non-Abelian) structures;

\item 
An example of `standard' supercategories was recently introduced in mathematical (or more specifically `categorified') physics, on the web's \PMlinkexternal{n-Category caf\'e's web site}{http://golem.ph.utexas.edu/category/2007/07/supercategories.html} under \textit{``Supercategories''}. This is a rather `simple' example of supercategories, albeit in a much more restricted sense as it still involves only the standard categorical homo-morphisms, homo-functors, and so on; it begins with a somewhat standard definiton of super-categories, or `super categories' from category theory, but then it becomes more interesting as it is being tailored to supersymmetry and extensions of `Lie' superalgebras, or superalgebroids, which are sometimes called graded `Lie' algebras that are thought to be relevant to quantum gravity (\cite{BGB2} and references cited therein). The following is an almost exact quote from the above n-Category cafe' s website posted mainly by Dr. Urs Schreiber:  
A \textit{supercategory} is a \textit{diagram} of the form: 
 $$\diamond  \diamond Id_C \diamond \textbf{C} \diamond \diamond s $$ 
in \textbf{Cat}--the category of categories and (homo-) functors between categories-- such that: 
 $$\diamond  \diamond \textsl{Id} \diamond \diamond Id_C \diamond \textbf{C} \diamond \textbf{C}\diamond \diamond s \diamond \diamond s = \diamond  \diamond Id_C \diamond Id_C  \diamond  \diamond \textsl{Id},$$
(where the `diamond' symbol should be replaced by the symbol `square', as in the original Dr. Urs Schreiber's postings.) 

 This specific instance is that of a supercategory which has only \textbf{one object}-- the above quoted superdiagram of diamonds, an arbitrary abstract category \textbf{C} (subject to all ETAC axioms), and the standard category identity (homo-) functor; it can be further specialized to the previously introduced concepts of \textit{supergroupoids} (also definable as crossed complexes of groupoids), and \textit{supergroups} (also definable as crossed modules of groups), which seem to be of great interest to mathematicians involved in `Categorified' mathematical physics or physical mathematics.) This was then continued with the following interesting example. ``What, in this sense, is a \textit{braided monoidal supercategory ?}''. Dr. Urs Schreiber, suggested the following answer: ``like an ordinary braided monoidal catgeory is a 3-category which in lowest degrees looks like the trivial 2-group, a braided monoidal supercategory is a 3-category which in lowest degree looks like the strict 2-group that comes from 
the crossed module $G(2)=(\diamond 2 \diamond \textsl{Id} \diamond 2)$''. Urs called this generalization of stabilization of n-categories, $G(2)$-\textit{stabilization}. Therefore, the claim would be that `braided monoidal supercategories come from $G(2)$-stabilized 3-categories, with $G(2)$ the above strict 2-group';

\item An \emph{organismic set} of order $n$ can be regarded either as a category of algebraic 
theories representing organismic sets of different orders $o \leq n$ or as a \emph{discrete topology} organismic supercategory of algebraic theories (or supercategory only with discrete topology, e.g. , a {\em class} of objects); 
\item Any `standard' topos with a (commutative) Heyting logic algebra as a subobject classifier is an example of
a commutative (and distributive) supercategory with the additional axioms to ETAC being those that
define the Heyting logic algebra;
\item The generalized $LM_n$ (\L ukasiewicz--Moisil) toposes are supercatgeories of 
{\em non-commutative}, algebraic $n$-valued logic diagrams that are subject to the axioms of {\em $LM_n$ algebras of 
$n$-valued logics};
\item $n$-categories are supercategories restricted to interpretations of the ETAC axioms;
\item An {\em organismic supercategory} is defined as a supercategory subject to the ETAC axioms
and also subject to the ETAS axiom of complete self--reproduction involving
$\pi$--entities ({\em viz}. L\"ofgren, 1968; \cite{Refs-13to26}); its objects are classes representing organisms
in terms of morphism (super) diagrams or equivalently as heterofunctors of organismic classes 
with variable topological structure;
\end{enumerate}

\begin{definition} 
\emph{Organismic Supercategories (\cite{Refs-13to26})}
An example of a class of supercategories interpreting such ETAS axioms as those stated above
was previously defined for organismic structures with different levels of complexity (\cite{Refs-13to26}); {\em organismic supercategories} were thus defined as {\em superstructure interpretations of ETAS} (including ETAC, as appropriate) in terms of triplets $\textbf{K} = (\textit{C}, \Pi,\textit{N})$, where \textit{C} is an arbitrary category (interpretation of ETAC axioms, formulas, etc.), $\Pi$ is a category of complete self--reproducing entities, $\pi$, (\cite{LO68}) subject to the negation of the axiom of restriction (for elements of sets):
$ \exists S: (S \neq \oslash) ~ and ~ \forall u: [u \in S) \Rightarrow \exists v: (v \in u)~ and ~( v \in S)]$, (which is known to be independent from the ordinary logico-mathematical and biological reasoning), 
and $\textit{N}$ is a category of non-atomic expressions, defined as follows.  
\end{definition} 

\begin{definition}
 
 An {\em atomically self--reproducing entity} is a unit class relation $u$ such that  $\pi \pi \left\langle \pi \right\rangle$, which means 
``$\pi$ stands in the relation $\pi$ to $\pi$'', $\pi \pi \left\langle \pi , \pi \right\rangle$, etc. 

 An expression that does not contain any such atomically self--reproducing entity is called a {\em non-atomic expression}.
\end{definition} 


\begin{thebibliography}{9}

\bibitem{Refs-13to26}
See references [13] to [26] in the \PMlinkname{Bibliography for Category Theory and Algebraic Topology}{CategoricalOntologyABibliographyOfCategoryTheory}

\bibitem{LW1}
W.F. Lawvere: 1963. Functorial Semantics of Algebraic Theories., \emph{Proc. Natl. Acad. Sci. USA}, \textbf{50}: 869--872.

\bibitem{LW2}
W. F. Lawvere: 1966. The Category of Categories as a Foundation for Mathematics. , In \emph{Proc. Conf. Categorical Algebra--La Jolla}, 1965, Eilenberg, S et al., eds. Springer --Verlag: Berlin, Heidelberg and New York, pp. 1--20.

\bibitem{LO68}
L. L\"ofgren: 1968. On Axiomatic Explanation of Complete Self--Reproduction. \emph{Bull. Math. Biophysics}, 
\textbf{30}: 317--348. 

\bibitem{BHS2}
R. Brown R, P.J. Higgins, and R. Sivera.: \textit{``Non--Abelian Algebraic Topology''} (2008).
\PMlinkexternal{PDF file}{http://www.bangor.ac.uk/mas010/nonab--t/partI010604.pdf} 
 
\bibitem{BGB2}
R. Brown, J. F. Glazebrook and I. C. Baianu: A categorical and higher dimensional algebra framework for complex systems and spacetime structures, \emph{Axiomathes} \textbf{17}:409--493.
(2007).

\bibitem{BM}
R. Brown and G. H. Mosa: Double algebroids and crossed modules of algebroids, University of Wales--Bangor, Maths Preprint, 1986.

\bibitem{BS}
R. Brown  and C.B. Spencer: Double groupoids and crossed modules, \emph{Cahiers Top. G\'eom.Diff.} \textbf{17} (1976), 343--362.

\bibitem{ICB04b}
I.C. Baianu: \L ukasiewicz--Topos Models of Neural Networks, Cell Genome and Interactome Nonlinear Dynamics). CERN Preprint EXT-2004-059. \textit{Health Physics and Radiation Effects} (June 29, 2004). 

\bibitem{BBGG1}
I.C. Baianu, Brown R., J. F. Glazebrook, and Georgescu G.: 2006, Complex Nonlinear Biodynamics in 
Categories, Higher Dimensional Algebra and \L ukasiewicz--Moisil Topos: Transformations of
Neuronal, Genetic and Neoplastic networks, \emph{Axiomathes} \textbf{16} Nos. 1--2, 65--122.

\bibitem{ICBm2}
I.C. Baianu and M. Marinescu: 1974, A Functorial Construction of \emph{\textbf{(M,R)}}-- Systems. \emph{Revue Roumaine de Mathematiques Pures et Appliquees} \textbf{19}: 388--391.

\bibitem{ICB6}
I.C. Baianu: 1977, A Logical Model of Genetic Activities in \L ukasiewicz Algebras: The Non--linear Theory. \emph{Bulletin of Mathematical Biophysics}, \textbf{39}: 249--258.

\bibitem{ICB2}
I.C. Baianu: 1987a, Computer Models and Automata Theory in Biology and Medicine.,  in M. Witten (ed.), 
\emph{Mathematical Models in Medicine}, vol. 7., Pergamon Press, New York, 1513--1577;
\PMlinkexternal{CERN Preprint No. EXT-2004-072}{http://doc.cern.ch//archive/electronic/other/ext/ext-2004-072.pdf}.
\end{thebibliography}
%%%%%
%%%%%
\end{document}
