\documentclass[12pt]{article}
\usepackage{pmmeta}
\pmcanonicalname{CalgebraHomomorphismsAreContinuous}
\pmcreated{2013-03-22 17:40:06}
\pmmodified{2013-03-22 17:40:06}
\pmowner{asteroid}{17536}
\pmmodifier{asteroid}{17536}
\pmtitle{$C^*$-algebra homomorphisms are continuous}
\pmrecord{14}{40105}
\pmprivacy{1}
\pmauthor{asteroid}{17536}
\pmtype{Theorem}
\pmcomment{trigger rebuild}
\pmclassification{msc}{81R15}
\pmclassification{msc}{46L05}
\pmsynonym{automatic continuity of $C^*$-homomorphisms}{CalgebraHomomorphismsAreContinuous}
\pmsynonym{homomorphisms of $C^*$-algebras are continuous}{CalgebraHomomorphismsAreContinuous}
%\pmkeywords{continuous linear mapping}
%\pmkeywords{$C^*$-algebra homomorphisms}
%\pmkeywords{$C_c (G) $}
%\pmkeywords{mapping continuity}
%\pmkeywords{C*-algebras and quantum compact groupoids}
\pmrelated{ContinuousLinearMapping}
\pmrelated{OperatorNorm}
\pmrelated{C_cG}
\pmrelated{UniformContinuityOverLocallyCompactQuantumGroupoids}
\pmrelated{CAlgebra}
\pmrelated{CAlgebra3}
\pmrelated{NormAndSpectralRadiusInCAlgebras}
\pmrelated{EquivalenceOfDefinitionsOfCAlgebra}
\pmrelated{GroupoidCConvolutionAlgebra}
\pmdefines{automatically continuous homomorphism of $C^*$--algebras}

% this is the default PlanetMath preamble.  as your knowledge
% of TeX increases, you will probably want to edit this, but
% it should be fine as is for beginners.

% almost certainly you want these
\usepackage{amssymb}
\usepackage{amsmath}
\usepackage{amsfonts}

% used for TeXing text within eps files
%\usepackage{psfrag}
% need this for including graphics (\includegraphics)
%\usepackage{graphicx}
% for neatly defining theorems and propositions
%\usepackage{amsthm}
% making logically defined graphics
%%%\usepackage{xypic}

% there are many more packages, add them here as you need them

% define commands here

\begin{document}
{\bf Theorem -} Let $\mathcal{A}, \mathcal{B}$ be \PMlinkname{$C^*$-algebras}{CAlgebra} and $f:\mathcal{A} \longrightarrow \mathcal{B}$ a *-homomorphism. Then $f$ is \PMlinkname{bounded}{ContinuousLinearMapping} and $\|f\| \leq 1$ (where $\|f\|$ is the \PMlinkname{norm}{OperatorNorm} of $f$ seen as a linear operator between the spaces $\mathcal{A}$ and $\mathcal{B}$).

For this reason it is often said that homomorphisms between $C^*$-algebras are \PMlinkname{automatically continuous}{ContinuousLinearMapping}.

{\bf Corollary -} A *-isomorphism between $C^*$-algebras is an \PMlinkname{isometric isomorphism}{IsometricIsomorphism}.\\
$\;$

{\bf \emph{Proof of Theorem :}} Let us first suppose that $\mathcal{A}$ and $\mathcal{B}$ have identity elements, both denoted by $e$.

We denote by $\sigma(x)$ and $R_{\sigma}(x)$ the spectrum and the spectral radius of an element $x \in \mathcal{A}$ or $\mathcal{B}$.

Let $a \in \mathcal{A}$ and $\lambda \in \mathbb{C}$. If $a- \lambda e$ is invertible in $\mathcal{A}$, then $f(a- \lambda e)$ is invertible in $\mathcal{B}$.  Thus,
\begin{displaymath}
\sigma(f(a)) \subseteq \sigma(a)\,.
\end{displaymath}
Hence $R_{\sigma}(f(a)) \leq R_{\sigma}(a)$ for every $a \in \mathcal{A}$. Therefore, by the result from \PMlinkname{this entry}{NormAndSpectralRadiusInCAlgebras},
\begin{displaymath}
\|f(a)\| = \sqrt{R_{\sigma}(f(a)^*f(a))} = \sqrt{R_{\sigma}(f(a^*a))} \leq \sqrt{R_{\sigma}(a^*a)}= \|a\|\,.
\end{displaymath}

We conclude that $f$ is \PMlinkescapetext{bounded} and $\|f\| \leq 1$.

If $\mathcal{A}$ or $\mathcal{B}$ do not have identity elements, we can consider their minimal unitizations, and the result follows from the above \PMlinkescapetext{argument}. $\square$

{\bf \emph{Proof of Corollary :}} This follows from the fact that $f^{-1}$ is also a *-homomorphism and therefore $\|f^{-1}(b)\|\leq \|b\|$ for every $b \in \mathcal{B}$. $\square$
%%%%%
%%%%%
\end{document}
