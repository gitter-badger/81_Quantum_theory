\documentclass[12pt]{article}
\usepackage{pmmeta}
\pmcanonicalname{FeynmanPathIntegral}
\pmcreated{2013-03-22 12:41:45}
\pmmodified{2013-03-22 12:41:45}
\pmowner{PrimeFan}{13766}
\pmmodifier{PrimeFan}{13766}
\pmtitle{Feynman path integral}
\pmrecord{19}{32976}
\pmprivacy{1}
\pmauthor{PrimeFan}{13766}
\pmtype{Definition}
\pmcomment{trigger rebuild}
\pmclassification{msc}{81S40}
%\pmkeywords{Feynman}
%\pmkeywords{path integral}
%\pmkeywords{quantum field theory}
%\pmkeywords{Feynman diagrams}
\pmrelated{LpSpace}
\pmrelated{RichardFeynman}

\endmetadata

\usepackage{amssymb}
\usepackage{amsmath}
\usepackage{amsfonts}

\newcommand{\RR}{\mathbb{R}}
\newcommand{\dotsint}{ \int \cdots}

\begin{document}
\PMlinkescapeword{argument}
\PMlinkescapeword{ranges}

A generalisation of multi-dimensional integral, written
\[
  \int \mathcal{D}\phi \;\mathrm{exp} \left (\mathcal{F} [\phi] \right )
\]
where $\phi$ ranges over some restricted set of functions from a measure space $X$ to some space with reasonably nice algebraic structure. The simplest example is the case where 
\[
  \phi \in L^2[X,\RR]
\]
and 
\[
  F[\phi] = -\pi \int_X \phi^2(x) d\mu(x)
\]
in which case it can be argued that the result is $1$. The argument is by analogy to the Gaussian integral
$\int_{\RR^n} dx_1\cdots dx_n e^{-\pi\sum x_j^2} \equiv 1$. Alas, one can absorb the $\pi$ into the measure on $X$.
Alternatively, following Pierre Cartier and others, one can use this analogy to {\em define} a measure on $L^2$
and proceed axiomatically.

One can bravely trudge onward and hope to come up with something, say \`a la Riemann integral, by partitioning $X$,
picking some representative of each partition, approximating the functional $F$ based on these
and calculating a  multi-dimensional integral as usual over the sample values of $\phi$. This leads to some integral
\[
  \dotsint d\phi(x_1) \cdots d\phi(x_n) e^{f(\phi(x_1),\ldots,\phi(x_n))}.
\]
One hopes that taking successively finer partitions of $X$ will give a sequence of integrals which converge on some nice limit. I believe Pierre Cartier has shown that this doesn't usually happen, except for the trivial kind of example given above.

The Feynman path integral was constructed as part of a re-formulation of \PMlinkescapetext{quantum field theory} by Richard Feynman, based on the sum-over-histories postulate of quantum mechanics, and can be thought of as an adaptation of Green's function methods for solving initial/boundary value problems. No appropriate measure has been found for this integral and attempts at pseudomeasures have given mixed results.

{\bf Remark:}
Note however that in solving quantum field theory problems one attacks the problem in the Feynman approach by
`dividing' it \emph{via} Feynman diagrams that are directly related to specific quantum interactions;
adding the contributions from such Feynman diagrams leads to high precision approximations to the final physical 
solution which is finite and physically meaningful, or observable.

\begin{thebibliography}{2}
\bibitem{hk} Hui-Hsiung Kuo, {\it Introduction to Stochastic Integration}. New York: Springer (2006): 250 - 253
\bibitem{jk} J. B. Keller \& D. W. McLaughlin, ``The Feynman Integral'' {\it Amer. Math. Monthly} {\bf 82} 5 (1975): 451 - 465
\end{thebibliography}
%%%%%
%%%%%
\end{document}
