\documentclass[12pt]{article}
\usepackage{pmmeta}
\pmcanonicalname{CanonicalQuantization}
\pmcreated{2013-03-22 15:53:34}
\pmmodified{2013-03-22 15:53:34}
\pmowner{bci1}{20947}
\pmmodifier{bci1}{20947}
\pmtitle{canonical quantization}
\pmrecord{18}{37894}
\pmprivacy{1}
\pmauthor{bci1}{20947}
\pmtype{Definition}
\pmcomment{trigger rebuild}
\pmclassification{msc}{81S10}
\pmclassification{msc}{53D50}
\pmclassification{msc}{46L65}
%\pmkeywords{quantization}
%\pmkeywords{quantum}
%\pmkeywords{Hamiltonian}
%\pmkeywords{Poisson algebra}
%\pmkeywords{correspondence principle}
\pmrelated{Quantization}
\pmrelated{PoissonBracket}
\pmrelated{HamiltonianOperatorOfAQuantumSystem}
\pmrelated{SchrodingerOperator}
\pmrelated{AsymptoticMorphismsAndWignerWeylMoyalQuantizationProcedures}
\pmdefines{operator substitution rule}
\pmdefines{operator ordering problem}

\usepackage{amssymb, amsmath, amsthm}
%%%\usepackage{xypic}

\newtheorem{thm}{Theorem}
\newtheorem{prop}[thm]{Proposition} 
\newtheorem{lemma}[thm]{Lemma}
\newtheorem{cor}[thm]{Corollary}

\theoremstyle{definition} 
\newtheorem{dfn}[thm]{Definition}

\theoremstyle{remark}
\newtheorem*{rmk}{Remark}
\newtheorem{ex}{Example}

\newcommand{\lie}{\mathcal{L}}
\newcommand{\pdiff}[2]{\frac{\partial #1}{\partial #2}}
\newcommand{\grad}{\nabla}
\newcommand{\reals}{\mathbb{R}}
\DeclareMathOperator{\Op}{Op}

\begin{document}
Canonical quantization is a method of relating, or associating, a classical system of the form $(T^*X, \omega, H)$, where $X$ is a manifold, $\omega$ is the canonical symplectic form on $T^*X$, with a (more complex) quantum system represented by $H \in C^\infty(X)$, where $H$ is the 
\PMlinkname{Hamiltonian operator}{HamiltonianOperatorOfAQuantumSystem}. Some of the early formulations of quantum mechanics used such quantization methods under the umbrella of the \emph{correspondence principle or postulate}.
The latter states that a correspondence exists between certain classical and quantum operators,
(such as the Hamiltonian operators) or algebras (such as Lie or Poisson (brackets)), with the
classical ones being in the real ($\mathbb{R}$) domain, and the quantum ones being in the complex ($\mathbb{C}$) domain.
Whereas all classical observables and states are specified only by real numbers, the 'wave' amplitudes in quantum
theories are represented by complex functions. \\

Let $(x^i, p_i)$ be a set of Darboux coordinates on $T^*X$. Then we may obtain from each coordinate function an operator on the Hilbert space $\mathcal{H} = L^2(X, \mu)$, consisting of functions on $X$ that are square-integrable with respect to some measure $\mu$, by the \emph{operator substitution} rule:
\begin{align}
x^i \mapsto \hat{x}^i &= x^i \cdot, \label{sub1}\\
p_i \mapsto \hat{p}_i &= -i \hbar \pdiff{}{x^i} \label{sub2},
\end{align}
where $x^i \cdot$ is the ``multiplication by $x^i$'' operator.  Using this rule, we may obtain operators from a larger class of functions.  For example,
\begin{enumerate}
\item $x^i x^j \mapsto \hat{x}^i \hat{x}^j = x^i x^j \cdot$,
\item $p_i p_j \mapsto \hat{p}_i \hat{p}_j = -\hbar^2 \pdiff{^2}{x^i x^j}$,
\item if $i \neq j$ then $x^i p_j \mapsto \hat{x}^i \hat{p}_j = -i \hbar x^i \pdiff{}{x^j}$.
\end{enumerate}

\begin{rmk}
The substitution rule creates an ambiguity for the function $x^i p_j$ when $i=j$, since $x^i p_j = p_j x^i$, whereas $\hat{x}^i \hat{p}_j \neq \hat{p}_j \hat{x}^i$.  This is the \emph{operator ordering} problem.  One possible solution is to choose 
\begin{equation*}x^i p_j \mapsto \frac{1}{2}\left(\hat{x}^i \hat{p}_j + \hat{p}_j \hat{x}^i\right),\end{equation*}
since this choice produces an operator that is self-adjoint and therefore corresponds to a physical observable.  More generally, there is a construction known as \emph{Weyl quantization} that uses Fourier transforms to extend the substitution rules (\ref{sub1})-(\ref{sub2}) to a map
\begin{align*}
C^\infty(T^*X) &\to \Op (\mathcal{H}) \\
f &\mapsto \hat{f}.
\end{align*}
\end{rmk}

\begin{rmk}
This procedure is called ``canonical'' because it preserves the canonical Poisson brackets.  In particular, we have that 
\begin{equation*}\frac{-i}{\hbar}[\hat{x}^i, \hat{p}_j] := \frac{-i}{\hbar}\left(\hat{x}^i\hat{p}_j - \hat{p}_j\hat{x}^i\right) = \delta^i_j,
\end{equation*}
which agrees with the Poisson bracket $\{ x^i, p_j \} = \delta^i_j$.
\end{rmk}

\begin{ex}
Let $X = \reals$.  The Hamiltonian function for a one-dimensional point particle with mass $m$ is
\begin{equation*}
H = \frac{p^2}{2m} + V(x),
\end{equation*}
where $V(x)$ is the potential energy.  Then, by operator substitution, we obtain the Hamiltonian operator
\begin{equation*}
\hat{H} = \frac{-\hbar^2}{2m} \frac{d^2}{dx^2} + V(x).
\end{equation*}
\end{ex}
%%%%%
%%%%%
\end{document}
