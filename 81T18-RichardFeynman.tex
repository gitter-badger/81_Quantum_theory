\documentclass[12pt]{article}
\usepackage{pmmeta}
\pmcanonicalname{RichardFeynman}
\pmcreated{2013-03-22 17:14:37}
\pmmodified{2013-03-22 17:14:37}
\pmowner{Mravinci}{12996}
\pmmodifier{Mravinci}{12996}
\pmtitle{Richard Feynman}
\pmrecord{16}{39573}
\pmprivacy{1}
\pmauthor{Mravinci}{12996}
\pmtype{Biography}
\pmcomment{trigger rebuild}
\pmclassification{msc}{81T18}
\pmclassification{msc}{01A60}
\pmsynonym{Richard Phillips Feynman}{RichardFeynman}
\pmsynonym{Richard P. Feynman}{RichardFeynman}
%\pmkeywords{QED}
%\pmkeywords{quantum electrodynamics}
%\pmkeywords{QCD}
%\pmkeywords{quantum chromodynamics}
%\pmkeywords{partons and quarks}
\pmrelated{FeynmanKacFormula}
\pmrelated{QEDInTheoreticalAndMathematicalPhysics}
\pmrelated{QuantumAutomataAndQuantumComputation2}
\pmrelated{FeynmannPathIntegral}

\endmetadata

% this is the default PlanetMath preamble.  as your knowledge
% of TeX increases, you will probably want to edit this, but
% it should be fine as is for beginners.

% almost certainly you want these
\usepackage{amssymb}
\usepackage{amsmath}
\usepackage{amsfonts}

% used for TeXing text within eps files
%\usepackage{psfrag}
% need this for including graphics (\includegraphics)
%\usepackage{graphicx}
% for neatly defining theorems and propositions
%\usepackage{amsthm}
% making logically defined graphics
%%%\usepackage{xypic}

% there are many more packages, add them here as you need them

% define commands here

\begin{document}
\PMlinkescapeword{project}

{\em Richard Phillips Feynman} (1918 - 1988) American theoretical physicist, best known for his contributions to quantum electrodynamics (QED). He won the 1965 Nobel Prize in Physics.

He was born in New York of Jewish immigrants. Barely in high school, young Richard was way ahead of his classmates, mastering calculus while they struggled with algebra. After high school he went to MIT and was appointed a Putnam Fellow upon graduation. Feynman obtained a \PMlinkescapetext{Ph.D.} from Princeton and married Arlene Greenbaum there. After working on the Manhattan Project during World War II, Feynman taught physics at Cornell and later CalTech. His most wide-ranging contributions are perhaps the Feynman diagram and the Feynman path integral, initially used in quantum mechanics, but now used widely in many other areas of mathematics and physics. {\em During WWII he contributed
to the Manhattan project by leading a team who carried out numerical computations related to the A-bomb development}.
Perhaps one of his more controversial contributions was in nuclear chromodynamics (QCD) where he postulated the existence of `partons'--hypothetical constituents of protons and neutrons in atomic nuclei-- that were later defined more precisely (by identifying the $SU(3)$ flavor symmetry of light `partons') by the 1969 Nobel laureate Murray Gell-Man, who also re-named them `quarks'. Richard Feynman and Murray Gell-Mann worked together on the vector structure of the weak interaction in physics, in parallel with the competing group of George Sudarshan and Robert Marshak; their work was soon to be fully developed after the \PMlinkexternal{``seminal discovery of parity violation by Chien-Shiung Wu, as suggested by Chen-Ning Yang and T. D. Lee''}{http://en.wikipedia.org/wiki/Murray_Gell-Mann}.
In 1981, in several public addresses, Richard Feynman suggested the possibility of developing 
\PMlinkname{quantum machines and quantum computers}{QuantumAutomataAndQuantumComputation2}.

Most physicists were impressed with Feynman's ability for applied mathematics. Feynman has \PMlinkname{Erd\H{o}s number}{ErdHosNumber} 4. With Murray Gell-Mann he published an article on the theory of Fermi interaction for {\it Phys. Rev. (2)} in 1958. Gell-Mann in turn wrote an article on gauge theories with Sheldon Glashow for another physics journal, and Glashow co-authored an article on mass formulas and mass inequalities with Coleman and Daniel Kleitman. Kleitman co-authored with Erd\H{o}s a paper ``On coloring graphs to maximize the proportion of multicolored $k$-edges'' in the {\it Journal of Combinatorial Theory} in 1968.

Famous for making jokes and pulling pranks, Feynman often claimed he wanted to memorize $\pi$ to 767 decimal \PMlinkescapetext{places} (now called the Feynman point in his honor). Feynman was honored on {\it \PMlinkescapetext{Star Trek}: The Next Generation} with a shuttlecraft named after him. More recently, the United States Postal Service honored Feynman with 37-cent stamps along with fellow Manhattan Project alum John von Neumann.

%%%%%
%%%%%
\end{document}
