\documentclass[12pt]{article}
\usepackage{pmmeta}
\pmcanonicalname{NoncommutativeGeometry}
\pmcreated{2013-03-22 18:13:53}
\pmmodified{2013-03-22 18:13:53}
\pmowner{bci1}{20947}
\pmmodifier{bci1}{20947}
\pmtitle{noncommutative geometry}
\pmrecord{31}{40818}
\pmprivacy{1}
\pmauthor{bci1}{20947}
\pmtype{Topic}
\pmcomment{trigger rebuild}
\pmclassification{msc}{81T75}
\pmsynonym{nonabelian algebraic topology}{NoncommutativeGeometry}
\pmsynonym{non-commutative geometry}{NoncommutativeGeometry}
\pmsynonym{non-Abelian geometry}{NoncommutativeGeometry}
\pmsynonym{anabelian geometry}{NoncommutativeGeometry}
\pmsynonym{non-Abelian topology (NAAT)}{NoncommutativeGeometry}
\pmsynonym{noncommutative topology}{NoncommutativeGeometry}
\pmsynonym{non-commutative topology}{NoncommutativeGeometry}
%\pmkeywords{C*-algebras}
%\pmkeywords{Quantum Gravity Theories}
%\pmkeywords{`deformation-quantization' of (commutative) spaces}
\pmrelated{CAlgebra}
\pmrelated{SpinGroup}
\pmrelated{FieldsMedal}
\pmrelated{CrafoordPrize}
\pmrelated{CAlgebra3}
\pmrelated{NuclearCAlgebra}
\pmrelated{QuantumGravityTheories}
\pmrelated{MathematicalProgrammesForDevelopingQuantumGravityTheories}
\pmrelated{QuantumGeometry}
\pmrelated{QuantumGeometry2}
\pmrelated{AlgebraicTopology}
\pmrelated{NoncommutativeTopology}
\pmdefines{`Geometry' of quantum spaces in terms of non-commutative algebras of functions and quantum operators}
\pmdefines{or `spectral (quantum) triples'}

\endmetadata

% this is the default PlanetMath preamble.  as your 
% almost certainly you want these
\usepackage{amssymb}
\usepackage{amsmath}
\usepackage{amsfonts}

% used for TeXing text within eps files
%\usepackage{psfrag}
% need this for including graphics (\includegraphics)
%\usepackage{graphicx}
% for neatly defining theorems and propositions
%\usepackage{amsthm}
% making logically defined graphics
%%%\usepackage{xypic}

% there are many more packages, add them here as you need them

% define commands here

\begin{document}
\section{Topic on Non-commutative Geometry (NCG)} 

 Noncommutative geometry utilizes non-Abelian (or nonabelian) methods for quantization of spaces through deformation to non-commutative 'spaces' (in fact {\em non-commutative} algebraic structures, or algebras of functions). 

 \emph{An alternative meaning is often given to noncommutative geometry (viz . A Connes et al.)}:  
that is, as a non-commutative `geometric' approach-- \emph{in the relativistic sense}-- to quantum gravity.

 A specific example due to A. Connes is the convolution $C^*$-algebra of (discrete) groups;
other examples are non-commutative $C^*$-algebras of operators defined on Hilbert spaces of
quantum operators and states. 

\subsection{Recent Developments in NCG} 
\begin{itemize}
\item The Royal Swedish Academy of Sciences has awarded the 2001 Crafoord Prize in mathematics
to Professor Alain Connes of the Institut des Hautes \'Etudes Scientifiques (IHES) and the
Coll\'ege de France, Paris, ``for his penetrating work on the theory of... (quantum)... operator algebras
and for having been a founder of \emph{noncommutative geometry}''.
(\PMlinkexternal{Crafoord Prize in 2001 in Noncommutative Geometry and Quantum Operator Algebras}{http://www.ams.org/notices/200105/comm-crafoord.pdf}).

Professor Alain Connes is also the 1983 recipient of the Field Medal. The following is a concise quote of his work
from the Crafoord Prize announcement in 2001: ``{\em Noncommutative geometry is a new field of mathematics, and
Connes played a decisive role in its creation. His work has also provided powerful new methods for treating renormalization theory and the standard model of quantum and particle physics...\PMlinkname{(SUSY)}{SpinGroup}... 
He has demonstrated that these new mathematical tools can be used for understanding and attacking the Riemann Hypothesis.}''

\item ``\emph{The Crafoord Prize prize consisted of a gold medal and US dollars 500,000. The Anna-Greta and Holger Crafoord Foundation was established in 1980 for promoting basic research in mathematics, astronomy, the biosciences
(particularly ecology), the geosciences, and polyarthritis (joint rheumatism)}''. Previous (`Nobel style'), Crafoord
Laureates in Mathematics were: Vladimir I. Arnold and Louis Nirenberg in 1982, Alexandre Grothendieck (who publicly declined the prize) and Pierre Deligne--who accepted the prize in 1988, and Simon Donaldson and Shing-Tung Yau (1994).
\end{itemize}


%%%%%
%%%%%
\end{document}
