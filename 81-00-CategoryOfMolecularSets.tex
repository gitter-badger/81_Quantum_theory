\documentclass[12pt]{article}
\usepackage{pmmeta}
\pmcanonicalname{CategoryOfMolecularSets}
\pmcreated{2013-03-22 18:16:02}
\pmmodified{2013-03-22 18:16:02}
\pmowner{bci1}{20947}
\pmmodifier{bci1}{20947}
\pmtitle{category of molecular sets}
\pmrecord{38}{40868}
\pmprivacy{1}
\pmauthor{bci1}{20947}
\pmtype{Topic}
\pmcomment{trigger rebuild}
\pmclassification{msc}{81-00}
\pmclassification{msc}{18E05}
\pmclassification{msc}{92B05}
\pmclassification{msc}{18D35}
\pmsynonym{class of molecular set variables and their transformations}{CategoryOfMolecularSets}
\pmsynonym{mcv-observable}{CategoryOfMolecularSets}
%\pmkeywords{molecular set theory}
%\pmkeywords{molecular set variables and their transformations}
\pmrelated{MolecularSetVariable}
\pmrelated{MolecularSetTheory}
\pmrelated{SupercategoriesOfComplexSystems}
\pmrelated{ComplexSystemsBiology}
\pmrelated{SupercategoryOfVariableMolecularSets}
\pmrelated{IndexOfCategories}
\pmrelated{CategoryOfQuantumAutomata}
\pmdefines{category of molecular sets}
\pmdefines{molecular set}
\pmdefines{molecular set mapping}
\pmdefines{states of molecular sets}
\pmdefines{uni-molecular chemical reaction}
\pmdefines{chemical transformation}
\pmdefines{molecular set variable}
\pmdefines{molecular transformation}
\pmdefines{$m.c.v.$ observable}
\pmdefines{mcv-observable}
\pmdefines{states of molecular se}

% this is the default PlanetMath preamble.  

% almost certainly you want these
\usepackage{amssymb}
\usepackage{amsmath}
\usepackage{amsfonts}

% used for TeXing text within eps files
%\usepackage{psfrag}
% need this for including graphics (\includegraphics)
%\usepackage{graphicx}
% for neatly defining theorems and propositions
%\usepackage{amsthm}
% making logically defined graphics
%%%\usepackage{xypic}

% there are many more packages, add them here as you need them

% define commands here
\usepackage{amsmath, amssymb, amsfonts, amsthm, amscd, latexsym}
%%\usepackage{xypic}
\usepackage[mathscr]{eucal}

\setlength{\textwidth}{6.5in}
%\setlength{\textwidth}{16cm}
\setlength{\textheight}{9.0in}
%\setlength{\textheight}{24cm}

\hoffset=-.75in     %%ps format
%\hoffset=-1.0in     %%hp format
\voffset=-.4in

\theoremstyle{plain}
\newtheorem{lemma}{Lemma}[section]
\newtheorem{proposition}{Proposition}[section]
\newtheorem{theorem}{Theorem}[section]
\newtheorem{corollary}{Corollary}[section]

\theoremstyle{definition}
\newtheorem{definition}{Definition}[section]
\newtheorem{example}{Example}[section]
%\theoremstyle{remark}
\newtheorem{remark}{Remark}[section]
\newtheorem*{notation}{Notation}
\newtheorem*{claim}{Claim}

\renewcommand{\thefootnote}{\ensuremath{\fnsymbol{footnote%%@
}}}
\numberwithin{equation}{section}

\newcommand{\Ad}{{\rm Ad}}
\newcommand{\Aut}{{\rm Aut}}
\newcommand{\Cl}{{\rm Cl}}
\newcommand{\Co}{{\rm Co}}
\newcommand{\DES}{{\rm DES}}
\newcommand{\Diff}{{\rm Diff}}
\newcommand{\Dom}{{\rm Dom}}
\newcommand{\Hol}{{\rm Hol}}
\newcommand{\Mon}{{\rm Mon}}
\newcommand{\Hom}{{\rm Hom}}
\newcommand{\Ker}{{\rm Ker}}
\newcommand{\Ind}{{\rm Ind}}
\newcommand{\IM}{{\rm Im}}
\newcommand{\Is}{{\rm Is}}
\newcommand{\ID}{{\rm id}}
\newcommand{\GL}{{\rm GL}}
\newcommand{\Iso}{{\rm Iso}}
\newcommand{\Sem}{{\rm Sem}}
\newcommand{\St}{{\rm St}}
\newcommand{\Sym}{{\rm Sym}}
\newcommand{\SU}{{\rm SU}}
\newcommand{\Tor}{{\rm Tor}}
\newcommand{\U}{{\rm U}}

\newcommand{\A}{\mathcal A}
\newcommand{\Ce}{\mathcal C}
\newcommand{\D}{\mathcal D}
\newcommand{\E}{\mathcal E}
\newcommand{\F}{\mathcal F}
\newcommand{\G}{\mathcal G}
\newcommand{\Q}{\mathcal Q}
\newcommand{\R}{\mathcal R}
\newcommand{\cS}{\mathcal S}
\newcommand{\cU}{\mathcal U}
\newcommand{\W}{\mathcal W}

\newcommand{\bA}{\mathbb{A}}
\newcommand{\bB}{\mathbb{B}}
\newcommand{\bC}{\mathbb{C}}
\newcommand{\bD}{\mathbb{D}}
\newcommand{\bE}{\mathbb{E}}
\newcommand{\bF}{\mathbb{F}}
\newcommand{\bG}{\mathbb{G}}
\newcommand{\bK}{\mathbb{K}}
\newcommand{\bM}{\mathbb{M}}
\newcommand{\bN}{\mathbb{N}}
\newcommand{\bO}{\mathbb{O}}
\newcommand{\bP}{\mathbb{P}}
\newcommand{\bR}{\mathbb{R}}
\newcommand{\bV}{\mathbb{V}}
\newcommand{\bZ}{\mathbb{Z}}

\newcommand{\bfE}{\mathbf{E}}
\newcommand{\bfX}{\mathbf{X}}
\newcommand{\bfY}{\mathbf{Y}}
\newcommand{\bfZ}{\mathbf{Z}}

\renewcommand{\O}{\Omega}
\renewcommand{\o}{\omega}
\newcommand{\vp}{\varphi}
\newcommand{\vep}{\varepsilon}

\newcommand{\diag}{{\rm diag}}
\newcommand{\grp}{{\mathbb G}}
\newcommand{\dgrp}{{\mathbb D}}
\newcommand{\desp}{{\mathbb D^{\rm{es}}}}
\newcommand{\Geod}{{\rm Geod}}
\newcommand{\geod}{{\rm geod}}
\newcommand{\hgr}{{\mathbb H}}
\newcommand{\mgr}{{\mathbb M}}
\newcommand{\ob}{{\rm Ob}}
\newcommand{\obg}{{\rm Ob(\mathbb G)}}
\newcommand{\obgp}{{\rm Ob(\mathbb G')}}
\newcommand{\obh}{{\rm Ob(\mathbb H)}}
\newcommand{\Osmooth}{{\Omega^{\infty}(X,*)}}
\newcommand{\ghomotop}{{\rho_2^{\square}}}
\newcommand{\gcalp}{{\mathbb G(\mathcal P)}}

\newcommand{\rf}{{R_{\mathcal F}}}
\newcommand{\glob}{{\rm glob}}
\newcommand{\loc}{{\rm loc}}
\newcommand{\TOP}{{\rm TOP}}

\newcommand{\wti}{\widetilde}
\newcommand{\what}{\widehat}

\renewcommand{\a}{\alpha}
\newcommand{\be}{\beta}
\newcommand{\ga}{\gamma}
\newcommand{\Ga}{\Gamma}
\newcommand{\de}{\delta}
\newcommand{\del}{\partial}
\newcommand{\ka}{\kappa}
\newcommand{\si}{\sigma}
\newcommand{\ta}{\tau}
\newcommand{\med}{\medbreak}
\newcommand{\medn}{\medbreak \noindent}
\newcommand{\bign}{\bigbreak \noindent}
\newcommand{\lra}{{\longrightarrow}}
\newcommand{\ra}{{\rightarrow}}
\newcommand{\rat}{{\rightarrowtail}}
\newcommand{\oset}[1]{\overset {#1}{\ra}}
\newcommand{\osetl}[1]{\overset {#1}{\lra}}
\newcommand{\hr}{{\hookrightarrow}}

\begin{document}
\subsection{Molecular sets as representations of chemical reactions}

A \emph{uni-molecular chemical reaction} is defined by the natural transformations    
$$\eta: h^A\longrightarrow h^B,$$  specified in the following commutative diagram representing molecular sets and their quantum transformations:
\begin{equation}
\def\labelstyle{\textstyle}
\xymatrix@M=0.1pc @=4pc{h^A(A) = Hom(A,A) \ar[r]^{\eta_{A}}
\ar[d]_{h^A(t)} &  h^B (A) = Hom(B,A)\ar[d]^{h^B (t)} \\  {h^A (B) =
Hom(A,B)}  \ar[r]_{\eta_{B}} & {h^B (B) = Hom(B,B)}},
\end{equation}

with the \emph{states of molecular sets} $A_u = a_1, \ldots, a_n$ and
$B_u = b_1, \ldots b_n$ being defined as the endomorphism sets $Hom(A,A)$ and $Hom(B,B)$, respectively. In general, \emph{molecular sets} $M_S$ are defined as finite sets whose elements are molecules; the \emph{molecules} are mathematically defined in terms of their molecular observables as specified next. In order to define molecular observables one needs to define first the concept of a molecular class variable or $m.c.v$.

 A \emph{molecular class variables} is defined as a family of molecular sets $[M_S]_{i \in I}$,
with $I$ being either an indexing set, or a proper class, that defines the variation range of the $m.c.v$}.
Most physical, chemical or biochemical applications require that $I$ is restricted to a finite set, (that is, without any sub-classes). A morphism, or molecular mapping,  $M_t: M_S \to M_S$ of molecular sets,  with $t \in T$ being real time values, is defined as a time-dependent mapping or function $M_S (t)$ also called a  \emph{molecular transformation}, $M_t$.

An \emph{$m.c.v.$ observable} of $B$, characterizing the products of chemical type ``B'' of a chemical reaction is defined as a morphism:

$$\gamma : Hom(B,B) \longrightarrow \Re ,$$
where $\Re$ is the set or field of real numbers. This mcv-observable is subject
to the following commutativity conditions:
\begin{equation}
\def\labelstyle{\textstyle}
  \xymatrix@M=0.1pc @=4pc{Hom(A,A) \ar[r]^{f}  \ar[d]_{e} & Hom(B,B)\ar[d]^{\gamma} \\  {Hom(A,A)}  \ar[r]_{\delta} & {R},}
\end{equation}~
with  $c: A^*_u  \longrightarrow  B^*_u$, and $A^*_u$, $B^*_u$  being, respectively,
specially prepared \emph{fields of states} of the molecular sets $A_u$, and $B_u$ within a measurement uncertainty range, $\Delta$, which is determined by Heisenberg's uncertainty relation, or the commutator of the observable operators involved, such as $[A^*, B^*]$, associated with the observable $A$ of molecular set $A_u$, and respectively, with the obssevable $B$ of molecular set $B_u$, in the case of a molecular set $A_u$ interacting with molecular set $B_u$.  

 With these concepts and preliminary data one can now define the category of molecular sets and their transformations 
as follows.

\subsection{Category of molecular sets and their transformations}
\begin{definition}
 The \emph{category of molecular sets} is defined as the category $C_M$ whose objects are molecular sets $M_S$ and whose morphisms are molecular transformations $M_t$. 
\end{definition}

\begin{remark}
 This is a mathematical representation of chemical reaction systems in terms of molecular sets that vary with time 
(or $msv$'s), and their transformations as a result of diffusion, collisions, and chemical reactions. 
\end{remark}

\begin{thebibliography}{9}

\bibitem{BAF60}
Bartholomay, A. F.: 1960. Molecular Set Theory. A mathematical representation for chemical reaction mechanisms. \emph{Bull. Math. Biophys.}, \textbf{22}: 285-307.

\bibitem{BAF65}
Bartholomay, A. F.: 1965. Molecular Set Theory: II. An aspect of biomathematical theory of sets., \emph{Bull. Math. Biophys.} \textbf{27}: 235-251.

\bibitem{BAF71}
Bartholomay, A.: 1971. Molecular Set Theory: III. The Wide-Sense Kinetics of Molecular Sets ., \emph{Bulletin of Mathematical Biophysics}, \textbf{33}: 355-372.

\bibitem{ICB2}
Baianu, I. C.: 1983, Natural Transformation Models in Molecular
Biology., in \emph{Proceedings of the SIAM Natl. Meet}., Denver,
CO.; Eprint at cogprints.org with No. 3675.

\bibitem{ICB2}
Baianu, I.C.: 1984, A Molecular-Set-Variable Model of Structural
and Regulatory Activities in Metabolic and Genetic Networks
\emph{FASEB Proceedings} \textbf{43}, 917.

\end{thebibliography}

%%%%%
%%%%%
\end{document}
