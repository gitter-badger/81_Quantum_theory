\documentclass[12pt]{article}
\usepackage{pmmeta}
\pmcanonicalname{LieAlgebroids}
\pmcreated{2013-03-22 18:13:47}
\pmmodified{2013-03-22 18:13:47}
\pmowner{bci1}{20947}
\pmmodifier{bci1}{20947}
\pmtitle{Lie algebroids}
\pmrecord{25}{40816}
\pmprivacy{1}
\pmauthor{bci1}{20947}
\pmtype{Topic}
\pmcomment{trigger rebuild}
\pmclassification{msc}{81P05}
\pmclassification{msc}{81R15}
\pmclassification{msc}{81R10}
\pmclassification{msc}{81R50}
\pmclassification{msc}{81R05}
\pmsynonym{extended quantum symmetry structures}{LieAlgebroids}
\pmsynonym{generalized double algebras}{LieAlgebroids}
%\pmkeywords{double algebras}
%\pmkeywords{extensions of Lie algebras}
%\pmkeywords{Lie groupoids}
%\pmkeywords{'Weinstein' groupoids}
%\pmkeywords{extended quantum symmetry structures}
\pmrelated{HamiltonianAlgebroids}
\pmrelated{TangentBundle}
\pmrelated{HamiltonianOperatorOfAQuantumSystem}
\pmrelated{Algebroids}
\pmrelated{LieSuperalgebra3}
\pmdefines{Lie algebroid}
\pmdefines{anchor}
\pmdefines{bundle map}

% this is the default PlanetMath preamble.  as your knowledge
% of TeX increases, you will probably want to edit this, but
% it should be fine as is for beginners.

% almost certainly you want these
\usepackage{amssymb}
\usepackage{amsmath}
\usepackage{amsfonts}

% define commands here
\usepackage{amsmath, amssymb, amsfonts, amsthm, amscd, latexsym, enumerate}
\usepackage{xypic, xspace}
\usepackage[mathscr]{eucal}
\usepackage[dvips]{graphicx}
\usepackage[curve]{xy}

\setlength{\textwidth}{6.5in}
%\setlength{\textwidth}{16cm}
\setlength{\textheight}{9.0in}
%\setlength{\textheight}{24cm}

\hoffset=-.75in     %%ps format
%\hoffset=-1.0in     %%hp format
\voffset=-.4in


\theoremstyle{plain}
\newtheorem{lemma}{Lemma}[section]
\newtheorem{proposition}{Proposition}[section]
\newtheorem{theorem}{Theorem}[section]
\newtheorem{corollary}{Corollary}[section]

\theoremstyle{definition}
\newtheorem{definition}{Definition}[section]
\newtheorem{example}{Example}[section]
%\theoremstyle{remark}
\newtheorem{remark}{Remark}[section]
\newtheorem*{notation}{Notation}
\newtheorem*{claim}{Claim}

\renewcommand{\thefootnote}{\ensuremath{\fnsymbol{footnote}}}
\numberwithin{equation}{section}

\newcommand{\Ad}{{\rm Ad}}
\newcommand{\Aut}{{\rm Aut}}
\newcommand{\Cl}{{\rm Cl}}
\newcommand{\Co}{{\rm Co}}
\newcommand{\DES}{{\rm DES}}
\newcommand{\Diff}{{\rm Diff}}
\newcommand{\Dom}{{\rm Dom}}
\newcommand{\Hol}{{\rm Hol}}
\newcommand{\Mon}{{\rm Mon}}
\newcommand{\Hom}{{\rm Hom}}
\newcommand{\Ker}{{\rm Ker}}
\newcommand{\Ind}{{\rm Ind}}
\newcommand{\IM}{{\rm Im}}
\newcommand{\Is}{{\rm Is}}
\newcommand{\ID}{{\rm id}}
\newcommand{\grpL}{{\rm GL}}
\newcommand{\Iso}{{\rm Iso}}
\newcommand{\rO}{{\rm O}}
\newcommand{\Sem}{{\rm Sem}}
\newcommand{\SL}{{\rm Sl}}
\newcommand{\St}{{\rm St}}
\newcommand{\Sym}{{\rm Sym}}
\newcommand{\Symb}{{\rm Symb}}
\newcommand{\SU}{{\rm SU}}
\newcommand{\Tor}{{\rm Tor}}
\newcommand{\U}{{\rm U}}

\newcommand{\A}{\mathcal A}
\newcommand{\Ce}{\mathcal C}
\newcommand{\D}{\mathcal D}
\newcommand{\E}{\mathcal E}
\newcommand{\F}{\mathcal F}
%\newcommand{\grp}{\mathcal G}
\renewcommand{\H}{\mathcal H}
\renewcommand{\cL}{\mathcal L}
\newcommand{\Q}{\mathcal Q}
\newcommand{\R}{\mathcal R}
\newcommand{\cS}{\mathcal S}
\newcommand{\cU}{\mathcal U}
\newcommand{\W}{\mathcal W}

\newcommand{\bA}{\mathbb{A}}
\newcommand{\bB}{\mathbb{B}}
\newcommand{\bC}{\mathbb{C}}
\newcommand{\bD}{\mathbb{D}}
\newcommand{\bE}{\mathbb{E}}
\newcommand{\bF}{\mathbb{F}}
\newcommand{\bG}{\mathbb{G}}
\newcommand{\bK}{\mathbb{K}}
\newcommand{\bM}{\mathbb{M}}
\newcommand{\bN}{\mathbb{N}}
\newcommand{\bO}{\mathbb{O}}
\newcommand{\bP}{\mathbb{P}}
\newcommand{\bR}{\mathbb{R}}
\newcommand{\bV}{\mathbb{V}}
\newcommand{\bZ}{\mathbb{Z}}

\newcommand{\bfE}{\mathbf{E}}
\newcommand{\bfX}{\mathbf{X}}
\newcommand{\bfY}{\mathbf{Y}}
\newcommand{\bfZ}{\mathbf{Z}}

\renewcommand{\O}{\Omega}
\renewcommand{\o}{\omega}
\newcommand{\vp}{\varphi}
\newcommand{\vep}{\varepsilon}

\newcommand{\diag}{{\rm diag}}
\newcommand{\grp}{\mathcal G}
\newcommand{\dgrp}{{\mathsf{D}}}
\newcommand{\desp}{{\mathsf{D}^{\rm{es}}}}
\newcommand{\grpeod}{{\rm Geod}}
%\newcommand{\grpeod}{{\rm geod}}
\newcommand{\hgr}{{\mathsf{H}}}
\newcommand{\mgr}{{\mathsf{M}}}
\newcommand{\ob}{{\rm Ob}}
\newcommand{\obg}{{\rm Ob(\mathsf{G)}}}
\newcommand{\obgp}{{\rm Ob(\mathsf{G}')}}
\newcommand{\obh}{{\rm Ob(\mathsf{H})}}
\newcommand{\Osmooth}{{\Omega^{\infty}(X,*)}}
\newcommand{\grphomotop}{{\rho_2^{\square}}}
\newcommand{\grpcalp}{{\mathsf{G}(\mathcal P)}}

\newcommand{\rf}{{R_{\mathcal F}}}
\newcommand{\grplob}{{\rm glob}}
\newcommand{\loc}{{\rm loc}}
\newcommand{\TOP}{{\rm TOP}}

\newcommand{\wti}{\widetilde}
\newcommand{\what}{\widehat}

\renewcommand{\a}{\alpha}
\newcommand{\be}{\beta}
\newcommand{\grpa}{\grpamma}
%\newcommand{\grpa}{\grpamma}
\newcommand{\de}{\delta}
\newcommand{\del}{\partial}
\newcommand{\ka}{\kappa}
\newcommand{\si}{\sigma}
\newcommand{\ta}{\tau}

\newcommand{\med}{\medbreak}
\newcommand{\medn}{\medbreak \noindent}
\newcommand{\bign}{\bigbreak \noindent}

\newcommand{\lra}{{\longrightarrow}}
\newcommand{\ra}{{\rightarrow}}
\newcommand{\rat}{{\rightarrowtail}}
\newcommand{\ovset}[1]{\overset {#1}{\ra}}
\newcommand{\ovsetl}[1]{\overset {#1}{\lra}}
\newcommand{\hr}{{\hookrightarrow}}

\newcommand{\<}{{\langle}}

%\newcommand{\>}{{\rangle}}
%\usepackage{geometry, amsmath,amssymb,latexsym,enumerate}
%%%\usepackage{xypic}

\def\baselinestretch{1.1}


\hyphenation{prod-ucts}

%\grpeometry{textwidth= 16 cm, textheight=21 cm}

\newcommand{\sqdiagram}[9]{$$ \diagram  #1  \rto^{#2} \dto_{#4}&
#3  \dto^{#5} \\ #6    \rto_{#7}  &  #8   \enddiagram
\eqno{\mbox{#9}}$$ }

\def\C{C^{\ast}}

\newcommand{\labto}[1]{\stackrel{#1}{\longrightarrow}}

%\newenvironment{proof}{\noindent {\bf Proof} }{ \hfill $\Box$
%{\mbox{}}
\newcommand{\midsqn}[1]{\ar@{}[dr]|{#1}}
\newcommand{\quadr}[4]
{\begin{pmatrix} & #1& \\[-1.1ex] #2 & & #3\\[-1.1ex]& #4&
 \end{pmatrix}}
\def\D{\mathsf{D}}

\begin{document}
\subsection{Topic on Lie algebroids}
 This is a topic entry on Lie algebroids that focuses on their quantum applications and extensions of current algebraic theories. 

\emph{Lie algebroids} generalize \emph{Lie algebras}, and in certain quantum systems they represent \emph{extended quantum (algebroid) symmetries}. One can think of a \emph{Lie algebroid} as generalizing the idea of a tangent bundle where the tangent space at a point is effectively the equivalence class of curves meeting at that point (thus
suggesting a groupoid approach), as well as serving as a site on which to study infinitesimal geometry (see, for example, ref. \cite{Mackenzie2005}). The formal definition of a \emph{Lie algebroid} is presented next. \\

\textbf{Definition 0.1}
Let $M$ be a manifold and let $\mathfrak X(M)$ denote the set of vector fields on $M$. Then, a 
\emph{Lie algebroid} over $M$ consists of a \emph{vector bundle $E \lra M$,
equipped with a Lie bracket $[~,~]$ on the space of sections $\gamma(E)$, 
and a bundle map $\Upsilon : E \lra TM$}, usually called the \emph{anchor}. 
Furthermore, there is an induced map $\Upsilon : \gamma (E) \lra \mathfrak X(M)$, 
which is required to be a map of Lie algebras, such that given sections $\a, \beta \in
\gamma(E)$ and a differentiable function $f$, the following
Leibniz rule is satisfied~:

\begin{equation}
[ \a, f \beta] = f [\a, \beta] + (\Upsilon (\a)) \beta~.
\end{equation}


\begin{example}  
A typical example of a Lie algebroid is obtained when $M$ is a Poisson
manifold and $E=T^*M$, that is $E$ is the cotangent bundle of $M$.
\end{example}

Now suppose we have a Lie groupoid $\mathsf{G}$:
\begin{equation}
r,s~:~ \xymatrix{ \mathsf{G} \ar@<1ex>[r]^r \ar[r]_s & \mathsf{G}^{(0)}}=M~.
\end{equation}
There is an associated Lie algebroid $\A = \A( \mathsf{G})$, which in the
guise of a vector bundle, it is the restriction to $M$ of the
bundle of tangent vectors along the fibers of $s$ (ie. the
$s$--vertical vector fields). Also, the space of sections $\gamma
(\A)$ can be identified with the space of $s$--vertical,
right--invariant vector fields $\mathfrak X^s_{inv} (\mathsf{G})$ which
can be seen to be closed under $[~,~]$, and the latter induces a
bracket operation on $\gamma (A)$ thus turning $\A$ into a Lie
algebroid. Subsequently, a Lie algebroid $\A$ is integrable if
there exists a Lie groupoid $\mathsf{G}$ inducing $\A$~.

\begin{remark}
Unlike Lie algebras that can be integrated to corresponding Lie groups, not all {\em Lie algebroids} are `smoothly integrable' to Lie groupoids; the subset of Lie groupoids that have corresponding Lie algebroids are sometimes called {\em `Weinstein groupoids'}.
\end{remark}

Note also the relation of the Lie algebroids to  Hamiltonian algebroids, also concerning recent developments in  (relativistic) quantum gravity theories.

\begin{thebibliography}{9}
\bibitem{Mackenzie2005}
K. C. H. Mackenzie: \emph{General Theory of Lie Groupoids and Lie
Algebroids}, London Math. Soc. Lecture Notes Series, \textbf{213},
Cambridge University Press: Cambridge,UK (2005).
\end{thebibliography}

%%%%%
%%%%%
\end{document}
