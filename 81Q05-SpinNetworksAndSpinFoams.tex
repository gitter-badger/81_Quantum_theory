\documentclass[12pt]{article}
\usepackage{pmmeta}
\pmcanonicalname{SpinNetworksAndSpinFoams}
\pmcreated{2013-03-22 18:14:34}
\pmmodified{2013-03-22 18:14:34}
\pmowner{bci1}{20947}
\pmmodifier{bci1}{20947}
\pmtitle{spin networks and spin foams}
\pmrecord{72}{40835}
\pmprivacy{1}
\pmauthor{bci1}{20947}
\pmtype{Topic}
\pmcomment{trigger rebuild}
\pmclassification{msc}{81Q05}
\pmclassification{msc}{81P05}
\pmclassification{msc}{81R15}
\pmclassification{msc}{81-00}
\pmsynonym{one-dimensional $CW$-complexes and category of $CW$ complexes}{SpinNetworksAndSpinFoams}
%\pmkeywords{spin networks and spin foams}
%\pmkeywords{$CW$ complex}
%\pmkeywords{connected CW complex}
%\pmkeywords{spin foams and category of $CW$  complexes}
\pmrelated{QuantumSpaceTimes}
\pmrelated{GroupRepresentation}
\pmrelated{MetricSuperfields}
\pmrelated{TwoDimensionalFourierTransforms}
\pmrelated{IrreducibleRepresentationsOfS_n}
\pmrelated{IrreducibleRepresentationsOfTheSpecialLinearGroupOverMathbbF_p}
\pmrelated{TheoremOnCWComplexApproximationOfQuantumStateSpacesInQAT}
\pmrelated{Spinor}
\pmrelated{Qua}
\pmdefines{formal spin network}
\pmdefines{formal spin foams}
\pmdefines{Lie group representation}

% this is the default PlanetMath preamble.  
% almost certainly you want these
\usepackage{amssymb}
\usepackage{amsmath}
\usepackage{amsfonts}
%\usepackage{amsthm}
% making logically defined graphics
%%%\usepackage{xypic}

% there are many more packages, add them here as you need them

% define commands here
\usepackage{amsmath, amssymb, amsfonts, amsthm, amscd, latexsym, enumerate}
\usepackage{xypic, xspace}
\usepackage[mathscr]{eucal}
\usepackage[dvips]{graphicx}
\usepackage[curve]{xy}

\setlength{\textwidth}{6.5in}
%\setlength{\textwidth}{16cm}
\setlength{\textheight}{9.0in}
%\setlength{\textheight}{24cm}

\hoffset=-.75in     %%ps format
%\hoffset=-1.0in     %%hp format
\voffset=-.4in


\theoremstyle{plain}
\newtheorem{lemma}{Lemma}[section]
\newtheorem{proposition}{Proposition}[section]
\newtheorem{theorem}{Theorem}[section]
\newtheorem{corollary}{Corollary}[section]

\theoremstyle{definition}
\newtheorem{definition}{Definition}[section]
\newtheorem{example}{Example}[section]
%\theoremstyle{remark}
\newtheorem{remark}{Remark}[section]
\newtheorem*{notation}{Notation}
\newtheorem*{claim}{Claim}

\renewcommand{\thefootnote}{\ensuremath{\fnsymbol{footnote}}}
\numberwithin{equation}{section}

\newcommand{\Ad}{{\rm Ad}}
\newcommand{\Aut}{{\rm Aut}}
\newcommand{\Cl}{{\rm Cl}}
\newcommand{\Co}{{\rm Co}}
\newcommand{\DES}{{\rm DES}}
\newcommand{\Diff}{{\rm Diff}}
\newcommand{\Dom}{{\rm Dom}}
\newcommand{\Hol}{{\rm Hol}}
\newcommand{\Mon}{{\rm Mon}}
\newcommand{\Hom}{{\rm Hom}}
\newcommand{\Ker}{{\rm Ker}}
\newcommand{\Ind}{{\rm Ind}}
\newcommand{\IM}{{\rm Im}}
\newcommand{\Is}{{\rm Is}}
\newcommand{\ID}{{\rm id}}
\newcommand{\grpL}{{\rm GL}}
\newcommand{\Iso}{{\rm Iso}}
\newcommand{\rO}{{\rm O}}
\newcommand{\Sem}{{\rm Sem}}
\newcommand{\SL}{{\rm Sl}}
\newcommand{\St}{{\rm St}}
\newcommand{\Sym}{{\rm Sym}}
\newcommand{\Symb}{{\rm Symb}}
\newcommand{\SU}{{\rm SU}}
\newcommand{\Tor}{{\rm Tor}}
\newcommand{\U}{{\rm U}}

\newcommand{\A}{\mathcal A}
\newcommand{\Ce}{\mathcal C}
\newcommand{\D}{\mathcal D}
\newcommand{\E}{\mathcal E}
\newcommand{\F}{\mathcal F}
%\newcommand{\grp}{\mathcal G}
\renewcommand{\H}{\mathcal H}
\renewcommand{\cL}{\mathcal L}
\newcommand{\Q}{\mathcal Q}
\newcommand{\R}{\mathcal R}
\newcommand{\cS}{\mathcal S}
\newcommand{\cU}{\mathcal U}
\newcommand{\W}{\mathcal W}

\newcommand{\bA}{\mathbb{A}}
\newcommand{\bB}{\mathbb{B}}
\newcommand{\bC}{\mathbb{C}}
\newcommand{\bD}{\mathbb{D}}
\newcommand{\bE}{\mathbb{E}}
\newcommand{\bF}{\mathbb{F}}
\newcommand{\bG}{\mathbb{G}}
\newcommand{\bK}{\mathbb{K}}
\newcommand{\bM}{\mathbb{M}}
\newcommand{\bN}{\mathbb{N}}
\newcommand{\bO}{\mathbb{O}}
\newcommand{\bP}{\mathbb{P}}
\newcommand{\bR}{\mathbb{R}}
\newcommand{\bV}{\mathbb{V}}
\newcommand{\bZ}{\mathbb{Z}}

\newcommand{\bfE}{\mathbf{E}}
\newcommand{\bfX}{\mathbf{X}}
\newcommand{\bfY}{\mathbf{Y}}
\newcommand{\bfZ}{\mathbf{Z}}

\renewcommand{\O}{\Omega}
\renewcommand{\o}{\omega}
\newcommand{\vp}{\varphi}
\newcommand{\vep}{\varepsilon}

\newcommand{\diag}{{\rm diag}}
\newcommand{\grp}{\mathcal G}
\newcommand{\dgrp}{{\mathsf{D}}}
\newcommand{\desp}{{\mathsf{D}^{\rm{es}}}}
\newcommand{\grpeod}{{\rm Geod}}
%\newcommand{\grpeod}{{\rm geod}}
\newcommand{\hgr}{{\mathsf{H}}}
\newcommand{\mgr}{{\mathsf{M}}}
\newcommand{\ob}{{\rm Ob}}
\newcommand{\obg}{{\rm Ob(\mathsf{G)}}}
\newcommand{\obgp}{{\rm Ob(\mathsf{G}')}}
\newcommand{\obh}{{\rm Ob(\mathsf{H})}}
\newcommand{\Osmooth}{{\Omega^{\infty}(X,*)}}
\newcommand{\grphomotop}{{\rho_2^{\square}}}
\newcommand{\grpcalp}{{\mathsf{G}(\mathcal P)}}

\newcommand{\rf}{{R_{\mathcal F}}}
\newcommand{\grplob}{{\rm glob}}
\newcommand{\loc}{{\rm loc}}
\newcommand{\TOP}{{\rm TOP}}

\newcommand{\wti}{\widetilde}
\newcommand{\what}{\widehat}

\renewcommand{\a}{\alpha}
\newcommand{\be}{\beta}
\newcommand{\grpa}{\grpamma}
%\newcommand{\grpa}{\grpamma}
\newcommand{\de}{\delta}
\newcommand{\del}{\partial}
\newcommand{\ka}{\kappa}
\newcommand{\si}{\sigma}
\newcommand{\ta}{\tau}

\newcommand{\med}{\medbreak}
\newcommand{\medn}{\medbreak \noindent}
\newcommand{\bign}{\bigbreak \noindent}

\newcommand{\lra}{{\longrightarrow}}
\newcommand{\ra}{{\rightarrow}}
\newcommand{\rat}{{\rightarrowtail}}
\newcommand{\ovset}[1]{\overset {#1}{\ra}}
\newcommand{\ovsetl}[1]{\overset {#1}{\lra}}
\newcommand{\hr}{{\hookrightarrow}}

\newcommand{\<}{{\langle}}

%\newcommand{\>}{{\rangle}}
%\usepackage{geometry, amsmath,amssymb,latexsym,enumerate}
%%%\usepackage{xypic}

\def\baselinestretch{1.1}


\hyphenation{prod-ucts}

%\grpeometry{textwidth= 16 cm, textheight=21 cm}

\newcommand{\sqdiagram}[9]{$$ \diagram  #1  \rto^{#2} \dto_{#4}&
#3  \dto^{#5} \\ #6    \rto_{#7}  &  #8   \enddiagram
\eqno{\mbox{#9}}$$ }

\def\C{C^{\ast}}

\newcommand{\labto}[1]{\stackrel{#1}{\longrightarrow}}

%\newenvironment{proof}{\noindent {\bf Proof} }{ \hfill $\Box$
%{\mbox{}}
\newcommand{\midsqn}[1]{\ar@{}[dr]|{#1}}
\newcommand{\quadr}[4]
{\begin{pmatrix} & #1& \\[-1.1ex] #2 & & #3\\[-1.1ex]& #4&
 \end{pmatrix}}
\def\D{\mathsf{D}}
\begin{document}
\begin{definition} \emph{spin networks} are one-dimensional $CW$ complexes consisting of quantum spin states of particles, defined by elements of Pauli matrices represented as vertices of a directed graph or network, and with the edges of the network representing the connections, or links, between such quantum spin states.
\end{definition}
\med

\begin{remark} \textbf{On current \emph{formal} definitions of spin networks.} 
  For quantum systems with known standard symmetry formal definitions of spin networks have also been reported in terms
of symmetry group representations. An example of such a formal definition in terms of Lie group
representations on Hilbert spaces of quantum states and operators is provided next.
\end{remark}
 
\begin{definition}
\emph{Spin networks} are formally defined here for quantum systems with `standard'* quantum symmetry in terms of Lie group ($G_L$) \PMlinkname{irreducible representations}{GroupRepresentation} on complex Hilbert spaces $\H$ of quantum states and observable operators; such representations are precisely defined by special group homomorphisms as follows.  
Consider $Re$ as a Lie group $G_L$, and also consider the complex Hilbert space $\H$ to be $B[\H]$, the group of bounded linear operators of $\H$ which have a bounded inverse, and more specifically to be $L^2(Re)$. 
Then, one defines the \emph{$G_L$-representation} as the group homomorphism $\rho: Re \to B[L^2(Re)]$ with 
$\rho(r): \left\{f(x)\right\} \mapsto f(r^{-1}x)$, where $r \in Re$ and $f(x) \in L^2(Re)$.

*The word `standard' is employed here with the meaning of the Standard Model of Physics (SUSY) which does \emph{not}
include either \emph{Quantum Gravity} or its extended quantum symmetries.
\end{definition}


\begin{definition} \emph{spin foams} are two-dimensional $CW$ complexes representing two local 
spin networks as described in \textbf{Definition 0.1} with quantum transitions between them; \emph{spin foams} are sometimes also represented by functors of spin networks considered as (small) categories (\emph{viz.} Baez and Dollan,1998a,b; \cite{BAJ-DJ98a, BAJ-DJ98b}).
\end{definition}

For the sake of completeness, let us recall here the following

\begin{definition} a \emph{$CW$ complex}, $X_c$ is a topological space which is the union of an expanding 
sequence of subspaces $X^n$ such that, inductively, $X^0$ is a discrete set of points called 
vertices and $X^{n+1}$ is the pushout obtained from $X^n$ by attaching disks $D^{n+1}$ along 
``attaching maps'' $j: S^n \rightarrow X^n$. Each resulting map $D^{n+1} \longrightarrow X$  is 
called a \emph{cell}.  The subspace $X^n$ is called the ``$n$-skeleton'' of X. 
\end{definition}

\textbf{An Example} of a $CW$ complex is a graph or `network' regarded as a one-dimensional $CW$ complex.

\textbf{Remark:}
Such `purely' topological definitions seem to miss much of the associated quantum operator algebraic structures
that are essential to the mathematical foundation of quantum theories; note however the first related entry
that addresses this important, algebraic question. 

\emph{Note.} 
  The concepts of {\em spin networks} and {\em spin foams} were recently developed in the context
of Mathematical Physics as part of the more general effort of attempting to formulate mathematically a concept of \emph{Quantum State Space} which is also applicable, or relates to \emph{Quantum Gravity} spacetimes. The {\em spin observable}-- which is fundamental in quantum theories-- has no corresponding concept in classical mechanics. (However, classical \emph{momenta} (both linear and angular) have corresponding quantum observable operators that are quite different in form, with their eigenvalues taking on different sets of values in Quantum Mechanics than the ones might expect from classical mechanics for the `corresponding' classical observables); the spin is an \emph{intrinsic} observable of all massive quantum `particles', such as electrons, protons, neutrons, atoms, as well as of all field quanta, such as photons, \emph{gravitons}, gluons, and so on; furthermore, every quantum `particle' has also associated with it a de Broglie wave, so that it cannot be realized, or `pictured', as any kind of classical `body'. This \emph{intrinsic}, spin observable, can also be understood as an \emph\emph{internal symmetry} of quantum particles, which in many cases can be understood in terms of `internal' symmetry group representations, such as the Dirac or Pauli matrices that are currently employed in quantum mechanics, Quantum Electrodynamics, QCD and QFT. There are thus \emph{fermion} (quantum) symmetries, quantum statistics, etc, for quantum particles with half-integer spin values (for massive particles such as electrons, protons,neutrons, quarks, nuclei with an \emph{odd} number of nucleons) and \emph{boson} (quantum) symmetries, statistics, etc., for quantum particles with integer spin values, such as $0, 1, 2, ..., n$, {where $n$ is usually thought to be less than $3$, for field quanta such as photons, gravitons, gluons, hypothetical Higgs bosons, etc).  

For massive quantum particles such as electrons, protons, neutrons, atoms, and so on, the spin property has been initially observed for atoms by applying a magnetic field as in the famous Stern-Gerlach experiment, (although the applied field may also be electric or gravitational, (see for example \cite{WH52})). All such spins interact with each other if the spin value is non-zero (i.e., generally, an integer, or half-integer) thus giving rise to ``spin networks'', which can be mathematically represented as in \textbf{Defintion 0.1} above; in the case of electrons, protons and neutrons such interactions are magnetic dipolar ones, and in an over-simplified, but not a physically accurate `picture', these are often thought of as `very tiny magnets--or magnetic dipoles--that line up, or flip up and down together, etc'. 

As a practical (and thus `intuitive', pictorial) example, the detection of all 
\PMlinkname{MRI (2D-FT) images}{TwoDimensionalFourierTransforms} employed in 
\PMlinkexternal{clinical medicine and biomedical research}{http://www.wpi.edu/Pubs/ETD/Available/etd-081707-080430/unrestricted/dbennett.pdf}, as well as all (multi-) Nuclear Magnetic Resonance (NMR) spectra employed in physical, chemical, biophyisical/biochemical/biomedical, polymer and agricultural research involves quantum transitions between spin networks or spin foams.  


\begin{thebibliography}{99}

\bibitem{WH52}
Werner Heisenberg. {\em The Physical Principles of Quantum Theory}. New York: Dover Publications, Inc.(1952), pp.39-47.

\bibitem{BF92}
F. W. Byron, Jr. and R. W. Fuller. {\em Mathematical Principles of Classical and Quantum Physics.}, New York: Dover Publications, Inc. (1992).

\bibitem{BAJ-DJ98a}
Baez, J. \& Dolan, J., 1998a, Higher-Dimensional Algebra III. n-Categories and the Algebra of Opetopes, in \emph{Advances in Mathematics}, \textbf{135}: 145-206.  

\bibitem{BAJ-DJ98b}
Baez, J. \& Dolan, J., 1998b, \emph{``Categorification'', Higher Category Theory, Contemporary Mathematics}, 
\textbf{230}, Providence: \emph{AMS}, 1-36. 

\bibitem{BAJ-DJ2k1}
Baez, J. \& Dolan, J., 2001, From Finite Sets to Feynman Diagrams, in \emph{Mathematics Unlimited -- 2001 and Beyond}, Berlin: Springer, pp. 29--50. 
\end{thebibliography}
%%%%%
%%%%%
\end{document}
