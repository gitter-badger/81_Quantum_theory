\documentclass[12pt]{article}
\usepackage{pmmeta}
\pmcanonicalname{GelfandTriple}
\pmcreated{2013-03-22 19:22:51}
\pmmodified{2013-03-22 19:22:51}
\pmowner{bci1}{20947}
\pmmodifier{bci1}{20947}
\pmtitle{Gel'fand triple}
\pmrecord{9}{42335}
\pmprivacy{1}
\pmauthor{bci1}{20947}
\pmtype{Topic}
\pmcomment{trigger rebuild}
\pmclassification{msc}{81P10}
\pmsynonym{rigged Hilbert space}{GelfandTriple}
%\pmkeywords{Riesz theorem}
%\pmkeywords{Gelfand triple}
%\pmkeywords{banach space}
%\pmkeywords{Hilbert space isomorphism}
\pmdefines{$\cS$}
\pmdefines{${\cS}^t$}
\pmdefines{injective bounded operator}
\pmdefines{$<m_x>$ operator}
\pmdefines{TVS}
\pmdefines{topological subvector space}
\pmdefines{canonical isomorphism}

\endmetadata

% almost certainly you want these
\usepackage{amssymb}
\usepackage{amsmath}
\usepackage{amsfonts}
\def\im{\operatorname{im}}
\def\ker{\operatorname{ker}}
% there are many more packages, add them here as you need 

% define commands here
\usepackage{amsmath, amssymb, amsfonts, amsthm, amscd, latexsym}
%%\usepackage{xypic}
\usepackage[mathscr]{eucal}
\theoremstyle{plain}
\newtheorem{lemma}{Lemma}[section]
\newtheorem{proposition}{Proposition}[section]
\newtheorem{theorem}{Theorem}[section]
\newtheorem{corollary}{Corollary}[section]
\theoremstyle{definition}
\newtheorem{definition}{Definition}[section]
\newtheorem{example}{Example}[section]
%\theoremstyle{remark}
\newtheorem{remark}{Remark}[section]
\newtheorem*{notation}{Notation}
\newtheorem*{claim}{Claim}

\renewcommand{\thefootnote}{\ensuremath{\fnsymbol{footnote%%@
}}}
\renewcommand{\H}{\mathcal H}

\numberwithin{equation}{section}

\newcommand{\Ad}{{\rm Ad}}
\newcommand{\Aut}{{\rm Aut}}
\newcommand{\Cl}{{\rm Cl}}
\newcommand{\Co}{{\rm Co}}
\newcommand{\DES}{{\rm DES}}
\newcommand{\Diff}{{\rm Diff}}
\newcommand{\Dom}{{\rm Dom}}
\newcommand{\Hol}{{\rm Hol}}
\newcommand{\Mon}{{\rm Mon}}
\newcommand{\Hom}{{\rm Hom}}
\newcommand{\Ker}{{\rm Ker}}
\newcommand{\Ind}{{\rm Ind}}
\newcommand{\IM}{{\rm Im}}
\newcommand{\Is}{{\rm Is}}
\newcommand{\ID}{{\rm id}}
\newcommand{\GL}{{\rm GL}}
\newcommand{\Iso}{{\rm Iso}}
\newcommand{\Sem}{{\rm Sem}}
\newcommand{\St}{{\rm St}}
\newcommand{\Sym}{{\rm Sym}}
\newcommand{\SU}{{\rm SU}}
\newcommand{\Tor}{{\rm Tor}}
\newcommand{\U}{{\rm U}}

\newcommand{\A}{\mathcal A}
\newcommand{\Ce}{\mathcal C}
\newcommand{\D}{\mathcal D}
\newcommand{\E}{\mathcal E}
\newcommand{\F}{\mathcal F}
\newcommand{\G}{\mathcal G}
\newcommand{\Q}{\mathcal Q}
\newcommand{\R}{\mathcal R}
\newcommand{\cS}{\mathcal S}
\newcommand{\cU}{\mathcal U}
\newcommand{\W}{\mathcal W}

\newcommand{\bA}{\mathbb{A}}
\newcommand{\bB}{\mathbb{B}}
\newcommand{\bC}{\mathbb{C}}
\newcommand{\bD}{\mathbb{D}}
\newcommand{\bE}{\mathbb{E}}
\newcommand{\bF}{\mathbb{F}}
\newcommand{\bG}{\mathbb{G}}
\newcommand{\bK}{\mathbb{K}}
\newcommand{\bM}{\mathbb{M}}
\newcommand{\bN}{\mathbb{N}}
\newcommand{\bO}{\mathbb{O}}
\newcommand{\bP}{\mathbb{P}}
\newcommand{\bR}{\mathbb{R}}
\newcommand{\bV}{\mathbb{V}}
\newcommand{\bZ}{\mathbb{Z}}

\newcommand{\bfE}{\mathbf{E}}
\newcommand{\bfX}{\mathbf{X}}
\newcommand{\bfY}{\mathbf{Y}}
\newcommand{\bfZ}{\mathbf{Z}}

\renewcommand{\O}{\Omega}
\renewcommand{\o}{\omega}
\newcommand{\vp}{\varphi}
\newcommand{\vep}{\varepsilon}

\newcommand{\diag}{{\rm diag}}
\newcommand{\grp}{{\mathbb G}}
\newcommand{\dgrp}{{\mathbb D}}
\newcommand{\desp}{{\mathbb D^{\rm{es}}}}
\newcommand{\Geod}{{\rm Geod}}
\newcommand{\geod}{{\rm geod}}
\newcommand{\hgr}{{\mathbb H}}
\newcommand{\mgr}{{\mathbb M}}
\newcommand{\ob}{{\rm Ob}}
\newcommand{\obg}{{\rm Ob(\mathbb G)}}
\newcommand{\obgp}{{\rm Ob(\mathbb G')}}
\newcommand{\obh}{{\rm Ob(\mathbb H)}}
\newcommand{\Osmooth}{{\Omega^{\infty}(X,*)}}
\newcommand{\ghomotop}{{\rho_2^{\square}}}
\newcommand{\gcalp}{{\mathbb G(\mathcal P)}}

\newcommand{\rf}{{R_{\mathcal F}}}
\newcommand{\glob}{{\rm glob}}
\newcommand{\loc}{{\rm loc}}
\newcommand{\TOP}{{\rm TOP}}

\newcommand{\wti}{\widetilde}
\newcommand{\what}{\widehat}

\renewcommand{\a}{\alpha}
\newcommand{\be}{\beta}
\newcommand{\ga}{\gamma}
\newcommand{\Ga}{\Gamma}
\newcommand{\de}{\delta}
\newcommand{\del}{\partial}
\newcommand{\ka}{\kappa}
\newcommand{\si}{\sigma}
\newcommand{\ta}{\tau}
\newcommand{\lra}{{\longrightarrow}}
\newcommand{\ra}{{\rightarrow}}
\newcommand{\rat}{{\rightarrowtail}}
\newcommand{\oset}[1]{\overset {#1}{\ra}}
\newcommand{\osetl}[1]{\overset {#1}{\lra}}
\newcommand{\hr}{{\hookrightarrow}}
\newcommand{\cok}{\operatorname{cok}}

\begin{document}
\section{Gel'fand (or Gelfand) triple}

The basic idea is to equip a separable Hilbert space $\H$ with a dense \emph{Topological Subvector Space ($TVS$)} of test functions in such a manner that the dual of the subspace of test functions $T_f$ enhances the Hilbert space $\H$ through embedding into a larger topological subvector space $T^*$; the elements of $T^*$ represent generalized eigenvectors for the continuous spectrum of normal-and possibly unbounded-linear operators $L$.  

One begins the Gelfand construction with a Banach space $B$, or the more general TVS considered above, and denote the dual TVS of $B$ as $B^*$ so that 
$j: B \rightarrow \H$ is an \emph{injective bounded operator} with dense image. One then considers a canonical isomorphism $i_c: \H \cong \H^*$ determined by the inner product given by the {\em Riesz theorem}. The Banach transpose, or dual of $j$ is the operator $j^*: \H^* \to B^*$. 
 

Then, one has a composition morphism $k: \H \rightarrow B^*$ as 
$\H  \to \H^* \to B^*$ , so that  $k = j^* \circ i_c$ that can be employed to define the Gelfand triple as follows.

\begin{definition}
A {\em Gelfand triple} is defined by the operator sequence:

$B \rightarrow \H \rightarrow B^*,$

 where $k$ is the composition morphism $k: \H \to B^*$ defined by $j^* \circ i_c$ 

\end{definition}

\subsection{Example:}
An interesting example of a Gel'fand triple which is preserved by the Fourier transform (FT) is obtained when $B = \cS (\R^n)$ --the Schwartz space, 
$B^* = \cS^t$-the space of tempered Schwartz distributions and 
$\H = L^2(\R^n)$.

\subsection{Acknowledgement}

The formalism utilized here followed the \PMlinkexternal{presentation by Urs Schreiber}{http://ncatlab.org/nlab/show/Gelfand+triple} of the Gel'fand triple at \PMlinkexternal{$nLab$- a `steered' wiki for Mathematics, Physical Mathematics and Philosophy}{http://ncatlab.org/nlab/show/HomePage}.

\subsection{References}

J.-P. Antoine. Dirac formalism and symmetry problems in quantum mechanics. I. General Dirac formalism. Journal of Mathematical Physics, 10(1):53--69, 1969.

N.Bogoliubov, A.Logunov, and I.Todorov. Introduction to Axiomatic Quantum Field Theory, chapter 1 Some Basic Concepts of Functional Analysis 4 The Space of States, pages 12--43, 113--128. Benjamin, Reading, Massachusetts, 1975.

R.de la Madrid. Quantum Mechanics in Rigged Hilbert Space Language. PhD thesis, Depertamento de Fisica Teorica Facultad de Ciencias. Universidad de Valladolid, 2001. (available here)

M.Gadella and F.Gomez. A unified mathematical formalism for the dirac formulation of quantum mechanics. Foundations of Physics, 32:815--869, 2002. (available here)

M.Gadella and F.Gomez. On the mathematical basis of the dirac formulation of quantum mechanics. International Journal of Theoretical Physics, 42:2225--2254, 2003.

M.Gadella and F.Gomez. Dirac formulation of quantum mechanics: Recent and new results. Reports on Mathematical Physics, 59:127--143, 2007.

I.M. Gel'fand and N.J. Vilenkin. Generalized Functions, vol. 4: Some Applications of Harmonic Analysis, volume4, chapter 2-4, pages 26--133. Academic Press, New York, 1964.

I. M. Gel'fand, A. G. Kostyucenko, Expansion in eigenfunctions of differential and other operators) (Russian), Dokl. Akad. Nauk SSSR (N.S.) 103 (1955), 349--352, MR73136

I. M. Gel'fand, N. Ja. Vilenkin, Generalized functions, vol. 4. Some applications of harmonic analysis. Equipped Hilbert spaces, Fizmatgiz, Moscow, 1961 MR146653, English transl. Acad. Press 1964 MR173945

Ciprian Foiash, D\'ecompositions int\'egrales des familles spectrales et 
semi--spectrales en op\'erateurs qui sortent de l'espace Hilbertien, Acta Sci. Math. Szeged 20, 1959, 117--155. MR115092

John Roberts, The Dirac bra and ket formalism, J. Mathematical Phys. 7 (1966), 1097--1104, MR216836, Rigged Hilbert spaces in quantum mechanics , Comm. Math. Physics 3, n. 2 (1966), 98--119, MR228243, euclid

J.E. Roberts. The Dirac bra and ket formalism. Journal of Mathematical Physics, 7(6):1097--1104, 1966.

A.R. Marlow. Unified Dirac--Von Neumann formulation of quantum mechanics. I. Mathematical theory. Journal of Mathematical Physics, 6:919--927, 1965.

E.Prugovecki. The bra and ket formalism in extended Hilbert space. J. Math. Phys., 14:1410--1422, 1973.

J.E. Roberts. Rigged Hilbert spaces in quantum mechanics. Commun. math. Phys., 3:98--119, 1966. 

Tjostheim. A note on the unified Dirac-Von Neumann formulation of quantum mechanics. Journal of Mathematical Physics, 16(4):766--767, 4 1975

%%%%%
%%%%%
\end{document}
